\documentclass[../DISSERTACAO_MAIN.tex]{subfiles}

\begin{document}


“QUANDO ME ENCONTRO NO CALOR DA LUTA”1

“Tudo que era sólido e estável se esfuma, tudo o
que era sagrado é profanado, e os homens são obrigados finalmente a encarar
com serenidade suas condições de existência e suas relações recíprocas”.
(Manifesto Comunista, Karl Marx)





“Tudo se repete continuamente. Ninguém aprende nada por não viver o suficiente para perceber o padrão”2. A citação poderia ser de George Santayana3, mas é dita por Marceline, rainha dos vampiros, no décimo segundo episódio da sétima temporada da animação Adventure Time. A frase que se tornou popular nas redes sociais nas vésperas das eleições de 2018 no Brasil já anunciava o fatídico resultado do segundo turno. Como previsto, o quadro atual tem confirmado que as preocupações acerca das consequências da ascensão de um candidato desqualificado, em todos os possíveis sentidos da palavra, cujo discurso traz à lembrança a imagem do mais famoso ditador da história, não eram vãs. Desde o golpe de 2016 tem se visto direitos trabalhistas sendo reduzidos, liberdades cerceadas e discursos de intolerância se propagando. Nas semanas posteriores às eleições, no entanto, a intensificação do processo de distanciamento da civilidade foi terrificante. Durante o período que compreendeu o primeiro até o segundo turno do processo eleitoral, foram realizados mais de cinquenta ataques contra povos indígenas, negros, mulheres e homossexuais externando discursos de ódio que há muito nos circundam.

Entretanto, ainda mais assustador, embora previsível, foram os notícias das semanas seguintes ao dia 28 de outubro, partindo da própria data em que Jair Bolsonaro foi eleito presidente: em menos de 20 minutos do resultado das eleições, a Aldeia Caarapó, em Pernambuco, foi atacada a tiros, a escola e o posto de saúde que atendiam ao povo Pakararu, também de Pernambuco, foram incendiados, em Curitiba, bares LGBT+s foram atacados, em Ponta Grossa, também no Paraná, um menino de oito anos foi morto por um disparo com arma de fogo feito em uma comemoração pró-Bolsonaro, em Niterói, militares são aplaudidos com gritos de “Viva a Ditadura”. Nos dias sequentes, notícias como o aceleramento dos processos da reforma da previdência e da implantação do Escola sem Partido, a indicação de envolvidos em corrupção para ministérios, o afastamento do Mercosul, o estremecimento das relações com a União Europeia, com a China, com o Egito e, previsivelmente, com outros países árabes, o prenúncio da guerra contra a Venezuela, a premiação dos serviços prestados por Sérgio Moro com um “superministério” e a saída dos médicos cubanos do Brasil davam uma amostra do que estaria por vir em um governo que oficialmente ainda nem havia começado. O cerceamento da mídia e o projeto da alteração da Lei Antiterrorismo, complementam o quadro de ruptura do Estado Democrático. E se ainda restasse alguma dúvida sobre o destino esperado para o país, a emissora de Sílvio Santos fez questão de retomar o slogan da ditadura militar: “Brasil, ame-o ou deixe-o”.

A memória do passado, a escrita da história lembra constantemente do que foram as ditaduras do século XX. Entretanto, nem todos os estudos sobre o totalitarismo, os alertas sobre o fascismo dos discursos e ações evitaram que o Brasil se deixasse seduzir pela extrema-direita. E não é um caso isolado. A subida de Trump ao poder foi acompanhada por notícias de partidos de extrema-direita ascendendo em outros países, sobretudo, nos europeus. Com representação nos governos, a parte da população que compactua com ideias que flertam com o fascismo tem se sentido livre para expressar seus pensamentos extremistas. Apesar disso, não é raro ouvir-se algo como: “São actos de desespero. Eles não têm força bastante. Não se vai passar nada” (TIAGO, 1997, p. 19).

Já no prefácio do livro Dialética do Esclarecimento, escrito em 1944, Adorno e Horkheimer explicam que o que se propõe na obra é “descobrir por que a humanidade, em vez de entrar em um estado verdadeiramente humano, está se afundando em uma nova espécie de barbárie” (1985, p.11). Do mesmo modo, pode-se perguntar hoje, em pleno século XXI, como o esclarecimento que deveria resultar num amadurecimento da humanidade, permitiu que massas apoiassem ditadores como Hitler, Salazar, Mussolini e Franco e, mais recentemente, nomes como Trump e Bolsonaro.

Na referida obra de Adorno e Horkheimer encontram-se algumas previsões pessimistas sobre a indústria cultural que não estão completamente longe da realidade. Os monopólios da mídia têm garantido que o poder de decidir o que se vê, o tipo de informação que deve ser transmitida, permaneça nas mãos dos detentores do capital. O prazer ‘se congela no aborrecimento’ durante o ócio, já que não se deve exigir esforço ou estimular o pensamento próprio. “Toda ligação lógica que pressuponha um esforço intelectual é escrupulosamente evitada” (ADORNO; HORKHEIMER, 1985, p. 128). Em A sociedade transparente, Gianni Vattimo explica que os mass media, que em tese deveriam ajudar a humanidade a se tornar mais iluminada por ter um acesso maior à informação, poderiam ser a concretização da ideia do Espírito Absoluto de Hegel, “uma perfeita autoconsciência de toda a humanidade, a coincidência entre aquilo que acontece, a história e a consciência do mundo” (1992, p. 12), na verdade, complexificam a compreensão daquilo que está ao redor ao reproduzir acontecimentos em tempo real.

A incapacidade em interpretar o mundo e os acontecimentos no seu entorno gerados por uma mídia que exige passividade, influenciou a forma como se deu o relacionamento dos indivíduos com o acesso à informação no ciberespaço. Numa era em que notícias e ideias se espalham numa velocidade assustadora graças à internet, logo toda sorte de pensamentos e ideias tomaram as redes sociais. Os youtubers que vêm, há cada dia mais, tomando o lugar dos jornalistas televisivos e da mídia impressa, tornaram-se os influenciadores das novas gerações. A pós-verdade toma o lugar da história e da ciência: a checagem de fatos e a pesquisa científica passam a ser mais vistas como opção do que como necessidade. O fato não importa mais, o que importa é a opinião. Num cenário desses, as teorias conspiratórias e equívocos históricos como terra planismo, movimento anti vacinas e nazismo de esquerda vêm sendo difundidos como verdades alternativas.  Além disso, as chamadas fake news se espalham de tal forma que, mesmo quando há uma responsabilização da fonte que as divulgou, o alastramento já não pode ser contido. O resultado do negacionismo tem sido observado no período de enfrentamento à pandemia de covid-19. Desde informações equivocadas acerca dos meios de transmissão e contenção do vírus, até a divulgação de tratamentos enganosos em detrimento da vacinação, único procedimento precoce efetivo, tem resultado em mortes que poderiam ter sido evitadas. 
 
A história humana se apresenta com avanços e recuos, repetição e inovação. Ao mesmo tempo que a ciência e tecnologia podem causar perplexidade quando se apresentam na forma da criação da internet, por exemplo, semelhante estupor pode ser causado pela atualidade das palavras de Adorno e Horkheimer quando esclarecem como se forma o pensamento fascista. Este movimento espiralado da história aparece nas narrativas distópicas. Ao se ler obras como 1984, Admirável Mundo Novo, Fahrenheit 451 e O conto da aia, por exemplo, deparar-se-á com universos onde a falência do humanismo leva sociedades tecnologicamente avançadas a mergulhar no obscurantismo. Não raro, o futuro previsto nestas narrativas traz características ou eventos que parecem alertas acerca do perigo em que as sociedades podem se encontrar caso não aprendam com os erros do passado.

Longe de imaginar o futuro através de narrativas distópicas, entretanto, esta tese analisa romances que reimaginam o passado. Acredita-se que, tão importante quanto olhar para o futuro, é ter os acontecimentos pregressos sempre em mente. Como mencionado no início deste texto, tomando emprestadas as palavras de George Santayana, se se esquece o passado, pode-se estar condenado a repeti-lo. 

Nota-se, pelas informações apresentadas no início desta introdução, que o Brasil está sob um Estado de exceção. Num período parecido com este, no ano de 1975, Chico Buarque voltou seus olhos para o que acontecia do outro lado do Atlântico. Em “Tanto mar”, canção que homenageia a Revolução dos Cravos, Chico pede que se mande para seu país “algum cheirinho de alecrim”. O compositor olha para a distância para pensar em formas de luta e transformação de uma realidade insatisfatória.

David Machado, João Tordo e João Ricardo Pedro vislumbram uma direção diferente: é o olhar para trás que irá pautar suas narrativas. Os romances analisados neste texto trazem o ponto de vista de três autores portugueses contemporâneos para o que foi o salazarismo em Portugal. Pretende-se, com esta tese, analisar o papel da literatura como preservação da memória, refletir acerca da importância  literária deixada pela geração antecedente e apontar para as formas de se escrever a pós-memória.

Muitos escritores portugueses da contemporaneidade tem buscado na história seus principais temas. Quer seja na reescrita de um passado remoto, quer seja em narrativas acerca de um período não tão distante como, por exemplo, as ficções que tomam o período salazarista como pano de fundo. A memória, oficial ou não, tem estado presente na literatura portuguesa através da escrita de autores como José Saramago, Lídia Jorge, Antônio Lobo Antunes e José Cardoso Pires. O relacionamento da literatura portuguesa com a história, todavia, é bem mais antigo: autores como Camões, Camilo Castelo Branco e Fernando Pessoa já haviam trazido para a tessitura literária os motivos da história.

O que há de novo, então? As narrativas que pretende-se analisar neste trabalho foram escritas por romancistas que não viveram a ditadura salazarista, mas ainda sim, tem uma relação de proximidade com este período da história. David Machado, João Ricardo Pedro e João Tordo fazem parte da chamada geração dos Novíssimos. São autores nascidos após a Revolução dos Cravos que voltam seu olhar para a memória dos seus pais e avós. Em Deixem falar as pedras, de David Machado, O teu rosto será o último, de João Ricardo Pedro, e Anatomia dos Mártires, de João Tordo, encontramos histórias que precisam ser contadas. Histórias de pessoas simples que fizeram parte da história. Nestas narrativas, cantam as musas os feitos dos pequenos homens: soldados rasos, camponesas e prisioneiros políticos.

Partindo da investigação do romance Anatomia dos Mártires que resultou na dissertação Os “apoderados da memória”: os herdeiros da ditadura salazarista em Anatomia dos Mártires, de João Tordo, ocorreu-se que assim como sucede com esta narrativa, haveria certamente outros textos dos Novíssimos nos quais a ditadura salazarista estaria presente como recuperação de uma memória geracional. A seleção dos romances Deixem falar as pedras, Anatomia dos Mártires e O teu rosto será o último entre outros textos se deve, principalmente, pelo tipo de narração que encontra-se nos textos, pelo período abrangido pelas narrativas e gerações das personagens e alguns outros aspectos.

A tese está organizada em cinco capítulos, a saber: “Tive, porém, que lembrar o meu passado”: no entorno do 25 de Abril, “Nem às paredes confesso”: a autocensura de Valdemar e as pedras que falam no romance de David Machado, “Ó gente da minha terra”: o martírio de uma camponesa segundo João Tordo, “Fiquei ao fim de tudo”: a escrita labiríntica do romance de João Ricardo Pedro e “É um canto a revelar toda uma vida”: como se escreve a herança.

Em “Tive, porém, que lembrar o meu passado”: no entorno do 25 de Abril, cumpre-se expor o que herdam os romances sobre os quais se falará. Para tal, apresentar-se-á os aspectos históricos e literários portugueses que influenciam no que é escrito pelos Novíssimos da literatura portuguesa. O capítulo toma como suporte teórico textos como: Mutações da literatura no século XXI, de Leyla Perrone-Moisés, Heterodoxia II, de Eduardo Lourenço, Espectros de Marx, de Jacques Derrida, Portugal: a Revolução e a Descolonização, de Maurício Paiva, A Revolução Portuguesa: o passado e o futuro, de Álvaro Cunhal, A Revolução dos Cravos, de Lincoln Secco, A Revolução Portuguesa, de Cláudio de Farias Augusto, entre outros. Além disso, para apresentar a pós-memória, utilizam-se, sobretudo, os textos The generation of postmemory: writing and visual culture after the holocaust, de Marianne Hirsch e Espaços da recordação, de Aleida Assmann. 

O segundo capítulo desta tese intitula-se “Nem às paredes confesso”: a autocensura de Valdemar e as pedras que falam no romance de David Machado. Nele analisar-se-á Deixem falar as pedras, de David Machado (2011). Neste romance concentrar-se-á, principalmente, no papel do narrador enquanto condutor das histórias que serão contadas, no testemunho como impulsionador desta narrativa e na relação que os signos presentes no texto tem com a literatura e a história portuguesa. Alguns dos textos em que se apoia para a análise deste romance são: O Espaço Biográfico, de Leonor Arfuch, Teoria do romance I: a estilística, de Mikhail Bakhtin, Memória e sociedade: lembrança dos velhos, de Ecléa Bosi, A memória, a história, o esquecimento, de Paul Ricoeur e A memória coletiva, de Maurice Halbwachs. 

Em “Ó gente da minha terra”: o martírio de Catarina Eufémia segundo João Tordo, abordar-se-á acerca do processo de conversão do assassinato de Catarina Eufémia pela GNR durante a ditadura salazarista em mito da esquerda portuguesa e, sucessivamente, mito literário, a partir do olhar do narrador de João Tordo em Anatomia dos Mártires (2011). A figura do narrador e uma escrita que se aproxima do ensaio funcionam como elementos fundamentais para a narrativa da pós-memória no romance. Alguns dos textos utilizados são: O narrador ensimesmado, de Maria Lúcia Dal Farra, Mitologias, de Roland Barthes,  A imaginação simbólica, de Gilbert Durand, A ficcionalização da História, de Márcia Valéria Zamboni Gobbi, Problemas da poética de Dostoiévski, de Mikhail Bakhtin e A alma e as formas, de Georg Lukács. 

O quarto capítulo, “Fiquei ao fim de tudo”: a escrita labiríntica do romance de João Ricardo Pedro, analisará O teu rosto será o último (2012). Neste romance onde acompanha-se a história de três gerações atravessadas pela repressão salazarista e pela guerra colonial, destaca-se um narrador heterodiegético, e uma escrita mosaica do trauma. São usados os textos: Fábulas da memória, de Lucette Valensi, Diante do extremo, de Tzvetan Todorov, A personagem de ficção, de Antonio Candido,  Espaço e Literatura: introdução à topoanálise, de Ozíris Borges Filho, E agora, José?, de Cardoso Pires, entre outros.

No quinto e último capítulo da tese, “É um canto a revelar toda uma vida”: como se escreve a herança, o que se propõe é uma reflexão e comparação entre os três romances. Deixem falar as pedras, Anatomia dos Mártires e O teu rosto será  último se encontram ao trazer a morte como elemento essencial no desenvolvimento das histórias narradas, o relacionamento intergeracional entre as personagens e a escrita da pós-memória do salazarismo. Utilizam-se os textos: O labirinto da saudade, de Eduardo Lourenço, História, Memória, Literatura, de Márcio Seligmann-Silva, O anjo da história, de Walter Benjamin, O arco e a lira, Octavio Paz, assim como outros já referenciados. 

São usados como arcabouço teórico nesta tese, sobretudo, textos acerca da memória como Espaços de recordação: formas e transformações da memória cultural, de Aleida Assmann, O espaço biográfico: dilemas da subjetividade contemporânea, de Leonor Arfuch, A memória coletiva, de Maurice Halbwachs, Lembrar escrever esquecer, de Jeanne Gagnebin, Tempo passado: cultura da memória e guinada subjetiva, de Beatriz Sarlo. Além disso, o texto traz também teorias acerca do romance, do romance português contemporâneo, teorias da narração e do narrador e material para uma contextualização histórica.

Os versos do fado que dá título a esta tese, o “Fado da herança”, canção que, em si, traz a relação estabelecida entre a geração novíssima, representada pela fadista Filipa Cardoso, e a anterior representada pelos compositores Alfredo Duarte e Jorge Fernando, rezam:

Porque me toca tão fundo, a sua voz
Sem regresso, no meu peito ela se esconde
Como um pássaro de luz voando a sós
A pulsar dentro de mim, não sei bem onde

É um canto a revelar toda uma vida
Que a escolheu para lhe dar todas as dores
E a pele que se arrepia em mim vencida
Por lhe sentir a tristeza e os desamores

Como a alma transparece, de que lei
É refém a minha alma presa à sua
E sem querer chorar, tão triste chorei
Como a chuva sem pedir, molhando a rua

Fecha as mãos e no seu rosto tenso e sério
A expressão dum fado intenso, quase um medo
Meu anseio é desvendar o seu mistério
E saber da sua voz o seu segredo.

A voz – uma voz cujo seu canto revela uma vida – toca tão fundo no peito que arrepia a pele e faz com que suas tristezas e seus desamores sejam sentidos por quem a escuta. Ao tocar tão profundamente quem a ouve, ela faz com que o desejo seja “desvendar o seu mistério”. O que move este trabalho é uma tentativa de desvendar o mistério das vozes que com seu canto revelam vidas perdidas e resgatadas no nebuloso período da ditadura salazarista. É a tentativa de compreender como os romances de David Machado, João Tordo e João Ricardo Pedro narram acerca de uma geração cujos traumas e cicatrizes não podem ser esquecidos e, neste processo explicitam as suas heranças históricas e literárias. 


\end{document}