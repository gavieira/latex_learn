\documentclass[../DISSERTACAO_MAIN.tex]{subfiles}

\begin{document}

1. “TIVE, PORÉM, QUE LEMBRAR O MEU PASSADO”1: NO ENTORNO DO 25 DE ABRIL


“Por que me toca tão fundo, a sua voz
Sem regresso, no meu peito ela se esconde
Como um pássaro de luz voando a sós
A pulsar dentro de mim, não sei bem onde”
(Fado da herança, Alfredo Duarte e Jorge Fernando)




A parte que lhe cabe. Aquilo que foi deixado para trás. A herança é uma lembrança contínua do paradoxo que é a existência humana. Herdar lembra a mortalidade, que o recebido hoje dos que já se foram, em breve será deixado para a posterioridade. Entretanto, herdar também lembra que a morte nunca é completa se deixa para trás os vestígios de uma existência. Herda-se casa, dinheiro, propriedades comerciais, nome de família. Herda-se também cor dos olhos, cabelos, estatura, formato do nariz, cor da pele. Herda-se dos progenitores, herda-se dos antepassados, herda-se de um povo, herdar-se de uma nação. 
Além dos bens materiais e das características físicas, pode-se tornar herdeiro da memória. Herda-se a história de um país, a oficial, aquela ensinada nas escolas, livros e nos museus. Herda-se ainda as memórias que não são oficiais. São histórias mais particulares, passadas e repassadas pelas gerações anteriores. Histórias de uma comunidade, de uma família e de avós e pais. São memórias que se contrastam com a história oficial por não trazer como protagonistas majestosos reis, poderosos governantes ou grandiosos exércitos. São memórias que se formam a partir das narrativas de soldados rasos, jovens camponesas, pequenos operários, empregados da venda da esquina.
Nesta tese lança-se um olhar sobre as narrativas de autores que, por sua vez, voltam-se na direção desses figurantes da história. A mudança no foco da câmera do rosto imponente do general montado em seu cavalo para mais atrás, para um rosto perdido na multidão, para o jovem soldado que mal conseguia carregar o peso da arma que portava, passa a ser a perspectiva da Nova História. É o olhar sobre e deste pequeno soldado que dará uma visão mais próxima do que foi a guerra. O jovem soldado não estará embebido em glória e sucesso a ponto de os olhares sobre si ocultarem qualquer ponto negativo, cada pequena rusga mal explicada a fim de manter seu glorioso nome intacto. 
No prefácio do seu livro Pessoas Extraordinárias, Hobsbawn diz que “essas pessoas constituem a maioria da raça humana. As discussões entre os historiadores sobre o quão importante são na história os indivíduos e suas decisões não dizem respeito a elas. Os escritos sobre tais indivíduos ausentes na história deixaram traços pouco significativos na narrativa macrohistórica” (2005, p. 7). Entretanto, é em tais indivíduos que pessoas comuns conseguem se reconhecer. Embora enquanto sujeitos suas histórias possam parecer sem importância, coletivamente, como Hobsbawm prossegue dizendo, estes homens e mulheres são os principais atores da História. 
O século XX é um período extremamente agitado da história humana. O século que trazia o discurso do progresso e avanço tecnológico e ansiava por mostrar o sucesso do sistema capitalista evidenciou o movimento espiralado da história. Após a passagem de duas grandes guerras devastadoras e o enfrentamento a ditaduras e governos totalitários em diversos pontos do planeta, os relatos ouvidos já não são mais de algum fulano perdido em um tempo sobre o qual seja difícil sequer imaginar, mas sim de avós,  pais, a tia do padeiro da esquina, o irmão desaparecido de uma avó. Romances e contos passaram a ser mais protagonizados por camponeses, operários e pessoas colocadas à margem da sociedade. São histórias de homens e mulheres que não estavam preocupados sobre qual legado deixariam para as futuras gerações. Não lhes interessava deixar seus nomes impressos nas páginas da história, mas, antes, fazer aquilo que, talvez por um breve momento, em meio a uma tomada rápida de decisão, consideravam a coisa certa a ser feita. 
Nos romances analisados nesta tese, nomeadamente, Deixem falar as pedras, Anatomia dos Mártires e O teu rosto será o último, as personagens são revivificadas por narradores que relatam a herança da memória recebida de seus pais e avós. São narradores que tentam, nas páginas de seus romances, dar vida aos seus antepassados que abdicaram de suas existências para que pudessem ser livres. São relatos que mostram como as ditaduras do século XX moldaram a forma como as famílias se constituíram. Os romances de David Machado, João Tordo e João Ricardo Pedro, autores portugueses nascidos após o período nebuloso do salazarismo em Portugal, trazem para as páginas da literatura portuguesa representações de histórias cuja verossimilhança é atestada pelos relatos e testemunhos dos sobreviventes da opressão do Estado Novo. São narrativas de homens moldados sob a sombra do medo, vidas ditadas pelo desconcerto do mundo. Em “Escrita e morte”, Eduardo Lourenço começa por dizer que “uma parte da nossa geração não se viveu enquanto se ia vivendo” (2020, p. 11). As personagens construídas por esses três ficcionistas novíssimos fazem parte dessa geração que teve sua existência paralisada e ‘sobreviveu à espera da “verdadeira vida”’ (2020, p. 11).  Dar corpo a eles é talvez legar uma segunda vida à essa geração que por longas décadas não pôde viver plenamente. 


1.1 O Portugal pós-revolução


Era a ditadura salazarista já parte do passado de Portugal quando foi perguntado a alguns escritores portugueses o motivo pelo qual a produção literária parecia ter diminuído  consideravelmente no período logo após a Revolução dos Cravos. A resposta de Saramago foi que não sobrou tempo para a escrita uma vez que o período “representou para todos a possibilidade de exercer uma política às claras, uma aprendizagem de vida coletiva em moldes totalmente diferentes” (GOMES, 1993, p. 125). De forma coincidente, Lídia Jorge responderá à pergunta afirmando que “houve um momento cívico com uma força tão grande, com uma turbulência tão grande, que os escritores sentiram que não queriam escrever como estavam a escrever até aí, pois não tinham feito a síntese necessária, para perseguir um novo caminho” (IBIDEM, p. 146) e, Teolinda Gersão dirá que “quando as experiências por que passam são apaixonantes e as mudanças que acontecem profundas, só depois é que se entra num período em que é possível criar, quando há já uma certa distância em relação ao passado” (IBIDEM, p. 159). No ensaio “Literatura e Revolução dos Cravos” escrito dois anos após o 25 de Abril, José Cardoso Pires diz, ao se referir à situação da literatura logo após a Revolução:
A partir dessa data o escritor não era mais o animal à margem ou o ornamento tolerado que uma Política dita do Espírito pretendera estrangular durante meio século. De todas as áreas culturais a literatura tinha sido a mais segregada pelo ódio fascista, agora dispunha de voz total, a que quisesse. E participava, recebia incentivos, apelos vários à intervenção (PIRES, 1977, p. 274).

Cardoso Pires segue dizendo, em seu texto, que, quando as narrativas esperadas pelo resto do mundo para esse período não chegaram, os jornais apontavam para uma “cicatriz do passado”, julgavam os portugueses “estonteados pela claridade, ao cabo de anos e anos de silêncio e de obscurantismo” (PIRES, 1977, p. 275) e, os portugueses, ‘sondavam as suas apreensões: a mão que se suspende e se adia, que não transcreve o momento, que o não transcreve, que o não perdura’ (IBIDEM, p. 276). As vozes silenciadas, ignoradas ou cifradas agora podiam se expressar, porém o período era um tempo de ação. Era preciso conhecer o novo Portugal que surgia antes que se pudesse voltar a escrever. Era preciso reconstruir Portugal. Além disso, era necessário distanciamento dos acontecimentos do 25 de Abril para o gesto criador. “Toda a inventiva, todas as filtragens da memória, associação, surtos do inconsciente, tudo quanto recria e torna significante a experiência em termos de descrição provém da capacidade de distanciamento em relação ao concreto ou ao fenómeno experimentado” (IBIDEM, p. 278), afirma Cardoso Pires. 
O 25 de Abril, movimento organizado por militares de baixa patente, “os capitães de abril”, pôde sair do planejamento graças ao apoio popular2. O povo desobedece ao que é pedido nos comunicados dos militares e sai à rua. Este ato muda a história portuguesa e reconfigura o caráter militar da sublevação em popular. Para todo o mundo fica exposto como a revolução que acontecia em Portugal pertencia ao povo. Talvez por esta característica Portugal tenha ganhado destaque a ponto da temática do salazarismo preencher páginas de romances estrangeiros como Trem Noturno para Lisboa, do escritor suíço Pascal Mercier3 e Afirma Pereira, do italiano Antonio Tabucchi. 
Após décadas sob a repressão do Estado Novo, a reconstrução de um país não seria um trabalho simples, sobretudo, um país em que se intencionava fraterno, justo e livre. O período posterior ao 25 de Abril é um momento de grandes mudanças no cenário político e econômico português. A verdade é que embora se houvesse a necessidade indiscutível de se romper com o salazarismo, não havia um plano completo sobre o que aconteceria à Portugal se a sublevação corresse bem. Os militares, segundo Lincoln Secco, queriam apenas três coisas: “pôr fim à ditadura, resgatar o prestígio das Forças Armadas e terminar a Guerra Colonial em África” (SECCO, 2005, p. 6). Era de suma importância repensar Portugal e sua política. Os seis governos provisórios que se seguiram à Revolução dos Cravos tiveram curtos períodos de duração marcados por tentativas de golpes e desacordos entre partidos. O fato do Movimento das Forças Armadas terem recebido o apoio  popular não facilitava a governança. O fato é que, como enfatizado por Claudio de Farias Augusto em A Revolução Portuguesa, era incompatível com a proposta revolucionária ter Portugal um governo sendo exercido por um conselho militar quando esperava-se um regime democrático pluripartidário (cf. AUGUSTO, 2011, p. 87-8). 
As questões que tinham de ser resolvidas em Portugal passavam pela problemática da economia. Portugal teve uma industrialização tardia. Enquanto outros países na Europa se industrializavam, Portugal estava muito mais preocupado com as questões relacionadas às colônias, sua fonte de exploração. Foi o esforço em manter o controle sobre estes territórios que fez o antigo Império Ultramarino gastar muito dinheiro com a manutenção do exército em territórios africanos, enquanto no continente europeu as mulheres e filhos de soldados eram privados de direitos básicos devido à falta de dinheiro4. É justamente o maior orgulho de Salazar, a dominação sobre povos africanos, que resultará em insatisfação do povo e entre oficiais de baixa patente, a massa das forças armadas, resultando na sublevação que levará o salazarismo ao seu término5.
Os movimentos revolucionários nos territórios africanos que em 1961 começavam, principal fator que leva à queda do Regime, convertem o que seria uma importante fonte de renda portuguesa em um gasto que Portugal não podia arcar. Além de uma já dita insatisfação por parte de soldados resultante de elementos materiais relacionados às suas presenças em Angola, Guiné-Bissau e Moçambique, o próprio contato direto de portugueses com outros povos por estarem em África  e com leituras relacionadas às teorias marxistas, foi o elemento intangível que levou à sublevação. Quando se assiste às entrevistas de Salgueiro Maia, quando procura-se informações acerca dos capitães que estiveram envolvidos no 25 de Abril e, é claro, as reivindicações, o programa do MFA para os governos que se seguiriam ao Regime Salazarista, nota-se que há consciência de classe nestes indivíduos, há vestígios e até mesmo citações das leituras que foram realizadas, leituras que englobam filosofia, sociologia, economia, etc. Muitos dos soldados eram jovens universitários e recém-saídos da universidade, onde certamente tiveram contato com leituras, à época, consideradas subversivas. Há, portanto, um conhecimento derivado de vivência e teoria como principal fundamento para a criação de uma consciência de classes e, consequentemente, mobilização. Observa-se que, embora o momento posterior à Revolução dos Cravos não tenha sido plenamente arquitetado, a sublevação é organizada com antecedência. Lincoln Secco aponta para quase um ano de preparação:
Em 9 de setembro de 1973 ocorreu, sob o pretexto de um simples churrasco, na casa de campo do capitão Dinis de Almeida, nos arredores de Évora, a primeira reunião de oficiais de baixo e médio escalão, com destaque para os capitães. Essa foi considerada a reunião de fundação do movimento dos capitães. Em 6 de outubro, outra reunião numa residência particular de Lisboa resultou numa circular lançada em 23 de outubro, que colocou Vasco Lourenço, Dinis de Almeida, Otelo Saraiva de Carvalho e Vitor Alves, dentre outros, à frente da coordenação do movimento (SECCO, 2005, p. 32).


Depois disso, uma reunião realizada em Cascais em março de 1974 aprova o documento “As forças armadas e a nação”. Tentando assumir a liderança, o general Spínola lança seu livro “Portugal e o Futuro” onde faz uma crítica mais moderada ao Estado Novo. Segundo Lincoln Secco, 
o general Spínola advogava “o reconhecimento dos povos à autodeterminação”e o “recurso à consulta popular”, uma “solução federativa” que contemplasse a independência progressiva das colônias, por meio da sua integração numa “comunidade lusíada”, com eleição democrática dos seus representantes. Os objetivos de Spínola eram moderados e constituíam uma alternativa conservadora. Pois se apresentavam como antídoto à “desagregação de Portugal pela via revolucionária” (SECCO, 2005, p. 34)6.

O período sequente ao dia 25 de abril de 1974 foi uma época de esperança e profundas mudanças em Portugal. A nacionalização de alguns dos setores básicos em Portugal, a reforma agrária e o processo de descolonização em África, foram algumas das conquistas do país que se reconstruía. O plano era um governo que aos poucos fosse se adequando ao modo de vida socialista. Augusto cita Melo Antunes que diz:
[…] há que programar como fazer a transição da economia de mercado, como fazer a transição do capitalismo para as formas coletivizadas da produção, sem perda das liberdades democráticas, sem cair no caos e sem perder as conquistas já feitas […]; há que dispor de um estudo forte, capaz de impor as medidas que permitam a acumulação necessária ao aumento da produção […]. [E afirmava ainda que] o povo tem que compreender bem o Programa que o MFA vai executar, que ele não se destina a alimentar a burguesia (AUGUSTO, 2011, p. 148).

A guinada à esquerda em Portugal atraiu olhares externos. Lincoln Secco afirma que “a imprensa internacional se alarmou com o que denominava a “bolchevização de Portugal”, o surgimento de um satélite de Moscou no extremo Ocidente da Europa” etc” (2004, p. 134). As eleições, entretanto, seguiram um caminho diferente do que parecia provável: o PCP fracassou. Secco atribui o ocorrido à “circunstância histórica de submissão à ideologia salazarista por quase cinquenta anos, ao poder local, à estrutura familiar conservadora, às tradições rurais etc.” (2004, p. 136). De todo modo, já no relatório escrito por Álvaro Cunhal em novembro de 1976 e publicado como A Revolução Portuguesa: o passado e o futuro, apontava-se para distanciamentos do projeto original da Revolução, como a oposição à reforma agrária e uma tentativa de destruir os projetos de nacionalizações de empresas ao se pregar que não seria benéfico para os trabalhadores e que, além disso, a iniciativa privada cuidaria melhor da gestão destas. Com um olhar já à frente do seu tempo, Cunhal indicava para “o agravamento e deterioração da situação económica e financeira” como um dos problemas centrais a serem resolvidos. Em caso contrário, o imperialismo, sempre atento, ofereceria ajuda financeira e, desta forma colocaria Portugal sob uma posição submissa (cf. CUNHAL, 1994, p. 319-24).
Na mesma publicação de Álvaro Cunhal, foi incluído um artigo do autor escrito 20 anos depois da Revolução dos Cravos. Neste texto, ele confirma as previsões que haviam sido feitas anos antes: as potencialidades portuguesas provenientes do alvorecer de Abril estavam, aos poucos, sendo liquidadas por governos, como o de Cavaco Silva e suas políticas de direita (cf. CUNHAL, 1994, p. 38-9).
Embora num primeiro momento a burguesia e as organizações políticas ligadas ao salazarismo tenham ficado ‘dispersas, desorganizadas e desnorteadas’, como aponta Maurício Paiva em Portugal: a revolução e a descolonização (cf. 2017, p. 37), Claudio de Farias Augusto cita, em seu livro A revolução portuguesa, uma entrevista de Vasco Lourenço concedida ao jornal O Público 25 anos depois da Revolução em que este lamenta a recuperação da direita (2011, p. 144). A verdade é que enquanto a esquerda se preocupava com suas diferenças, o outro lado preparava seu retorno.
Há, pouco depois do 25 de Abril, um forte interesse do Estado português em fazer parte da Europa. Uma Europa, como aponta Lincoln Secco, “Ocidental, liberal, capitalista e, de preferência, social-democrata” (2004, p.194). Em Pela mão de Alice, Boaventura de Sousa Santos chama a década de 80, período em que Portugal entra para a Comunidade Económica Europeia, de “década do pós-marxismo” (1996, p. 29). Ele afirma:
A ascensão de partidos conservadores na Europa e nos EUA; o isolamento progressivo dos partidos comunistas e a descaracterização política dos partidos socialistas; a transnacionalização da economia e a sujeição férrea dos países periféricos e semi-periféricos às exigências do capitalismo multinacional e das suas instituições de suporte, o Banco Mundial e o Fundo Monetário Internacional; a consagração mundial da lógica económica capitalista sob a forma neoliberal e a consequente apologia do mercado, da livre iniciativa, do Estado mínimo, e da mercantilização das relações sociais; o fortalecimento sem precedentes da cultura de massas e a celebração nela de estilos de vida e de imaginários sociais individualistas, privatistas e consumistas, militantemente relapsos a pensar a possibilidade de uma sociedade alternativa ao capitalismo ou sequer a exercitar a solidariedade, a compaixão ou a revolta perante a injustiça social;  a queda consentida de governos de orientação socialista às mãos do jogo democrático antes julgado burguês na Nicarágua, em Cabo Verde e outros países; e, finalmente, o rotundo e quase inacreditável colapso dos regimes comunistas no Leste europeu – todos estes factores foram convergindo para transformar o marxismo, aos olhos de muitos, como pouco mais que um anacronismo (IBIDEM, p. 29). 
O desvio na orientação de um país que havia passado por uma revolução à esquerda e se converte em um Portugal neoliberal, aparece nas narrativas examinadas nesta tese. Segundo apontado por Marcuse, o retorno ao passado revela que há insuficiência no presente (cf. MARCUSE, 1975, p. 38-9). Há, nesses textos, uma espécie de desilusão com os rumos que foram tomados pela história. Publicados em 2011 e 2012, os romances Deixem falar as pedras, Anatomia dos Mártires e O teu rosto será o último são escritos e retratam como tempo presente o segundo mandato de Cavaco Silva, representante da centro-direita. O desemprego, por exemplo, que marca essa época em Portugal, aparece representado nos textos7. A desilusão ilustrada nas três narrativas é comprovada pelos dados estatísticos. Segundo apontado por Claudio de Farias Augusto:
para reafirmar o desencanto, podemos recorrer aos registros da rotina eleitoral portuguesa, pautada no voto voluntário. Ela tem registrado um progressivo desinteresse da população pelas eleições (sobretudo as nacionais), chegando o índice de abstenção a ultrapassar os 50% nas eleições presidenciais de 2011, quando o presidente Aníbal Cavaco e Silva (de centro-direita) foi reeleito com o menor número de votos desde 1976 – quando, no fim da aurora do “novo futuro”, mais de 75% dos eleitores foram às urnas (2011, p. 171).

Recorrer ao período revolucionário nesses romances é uma tentativa de recuperação do espírito que acompanhou o 25 de Abril. A falta de esperança quanto ao futuro, leva ao questionamento acerca do que foi feito no passado e no que pode ser feito para, não recuperar tal qual aconteceu, mas trazer de volta os ideias e as vibrações que tornaram possível que a história se escrevesse com movimentos de transformação pautadas em ideias de liberdade, fraternidade e igualdade. 


1.2 Quando a literatura reescreve a história


Na Poética, de Aristóteles, é dito que “a função do poeta não é contar o que aconteceu mas aquilo que poderia acontecer”. No texto, Aristóteles ainda apresenta elementos que, passados mais de dois mil anos, continuam a servir para estabelecer uma distinção entre a escrita literária e a escrita histórica. Seu texto reza:
O historiador e o poeta não diferem pelo facto de um escrever em prosa e o outro em verso (se tivéssemos posto em verso a obra de Heródoto, com verso ou sem verso ela não perderia absolutamente nada o seu carácter de História). Diferem é pelo facto de um relatar o que aconteceu e o outro o que poderia acontecer. Portanto, a poesia é mais filosófica e tem um carácter mais elevado do que a História. É que a poesia expressa o universal, a História o particular. O universal é aquilo que certa pessoa dirá ou fará, de acordo com a verossimilhança ou a necessidade, e é isso que a poesia procura representar, atribuindo, depois, nomes às personagens. O particular é, por exemplo, o que fez Alcibíades ou o que lhe aconteceu (ARISTÓTELES, 2011, p. 54).  

Sem deixar ainda o mundo grego antigo, e embora uma relate o que aconteceu e, a outra, o que poderia ter acontecido, Literatura e História estão ligadas por uma mãe comum: a memória. Avançando-se ao longo do tempo, é possível encontrar diversas narrativas onde história e ficção se cruzam e muitos narradores que utilizarão acontecimentos reais como elementos para a escrita ficcional. As pessoas que foram atores da história, os lugares e tempo em que os eventos ocorreram e os acontecimentos ajudam a dar vida à personagens, espaço, tempo e enredo nas ficções.
O uso da história como matéria-prima para a escrita literária, portanto, apresenta-se como um dos elementos que não é novidade no campo narrativo. Em uma entrevista concedida ao Jornal de Letras e citada por Carlos Reis em História Crítica da Literatura Portuguesa vol. IX, José Saramago diz que a tentação que, enquanto escritor, tem sobre a História é a de tentar corrigi-la, “substituir o que foi pelo que poderia ter sido” e ainda aponta, na sequência:
Duas serão as atitudes possíveis do romancista que escolheu, para a sua ficção, os caminhos da História: uma, discreta e respeitosa, consistirá em reproduzir ponto por ponto os factos conhecidos, sendo a ficção mera servidora duma fidelidade que se quer inatacável; a outra, ousada, leva-lo-á a entretecer dados históricos não mais que suficientes num tecido ficcional que se manterá predominante. Porém, estes dois vastos mundos, à primeira vista inconciliáveis, podem vir a ser harmonizados na instância narradora (2006, p. 322-3).

Ao mesmo tempo que se encontra na Literatura essas possibilidades do tratamento da história como matéria ficcional apontadas por Saramago, há profundas mudanças ocorridas na escrita da história. A Nova História volta o seu olhar para as personagens deixadas de lado pela história tradicional. No capítulo introdutório de A escrita da história, Peter Burke diz que “a nova história é a história escrita como uma reação deliberada contra o “paradigma” tradicional” (2011, p. 10). Para chegar a uma aproximação do que seria o conceito dessa nova história, Burke opta por defini-la pelo que não é apontando para seis pontos. No terceiro ponto, o historiador britânico opõe a perspectiva que parte do feito dos grandes homens na história tradicional com um olhar que vem de baixo pela nova história:
Por outro lado, vários novos historiadores estão preocupados com a história vista de baixo; em outras palavras, com as opiniões das pessoas comuns e com sua experiência da mudança social. A história da cultura popular tem recebido bastante atenção. Os historiadores da Igreja estão começando a estudar sua história vista tanto de baixo, quanto de cima. Os historiadores intelectuais também têm deslocado sua atenção dos grandes livros ou das grandes ideias – seu equivalente aos grandes homens –  para a história das mentalidades coletivas ou para a história dos discursos ou “linguagens”, a linguagem da escolástica, por exemplo, ou a linguagem forense (BURKE, 2011, p. 13)8.

Essa perspectiva nova da história se funde à escrita literária: nos romances contemporâneos que trazem a história com força motriz, encontra-se a narrativa da vida de homens e mulheres considerados pequenos, personagens que, como afirma Hobsbawm na já citada introdução à Pessoas extraordinárias, não influenciam na macro história. A novidade de se escrever sobre os pequenos homens e mulheres, entretanto, foi característica de uma literatura que se pretende espelho da sociedade, o neorrealismo. E esse olhar sobre as pessoas comuns permanecerá presente na escritura da contemporaneidade composta, em Portugal, por nomes como José Saramago, António Lobo Antunes, Lídia Jorge e José Cardoso Pires, alguns dos autores que movimentaram e (continuam a movimentar!), o cenário literário português. 
Embora os fatos históricos sejam importantes enquanto temática, o foco destas narrativas tende a ser no fator humano. O desvio do olhar é para os seres humanos que são afetados, que sofrem as consequências do período histórico. Esse elemento tem tido destaque ao se trazer para as páginas de romances e outros tipos de textos literários a temática das guerras e ditaduras do século XX. Em Diante do extremo, Todorov aponta para uma questão relacionada à escrita da história que pode indicar também o modo como se configura a escrita literária nesses casos: 
Os autores não aspiram a refazer a História; esta já se encontra estabelecida, mas – assim como o fazem os heróis – não se ocupa dos indivíduos. Ora, eles escolheram focar justamente nos detalhes e nos indivíduos. “Não estamos escrevendo a História. Falamos da Memória”(2012, p. 40). 
Ao tratar da história recente, de experiências cujos personagens que os vivenciaram continuam a andar sobre a terra, o testemunho torna-se também referência para a escrita ficcional. Tanto a história, quanto a literatura, lidam, neste caso, com a memória e, consequentemente, com o que é dúbio e incerto. Além disso, ambas as disciplinas se exprimem através das palavras e estas não podem dizer tal qual foi a experiência. Talvez seja possível dizer que as palavras traduzem a vivência. Assim como ocorre no trabalho da tradução, o resultado nunca será exatamente o que foi dito, porém, uma aproximação possível. 
Há que se lembrar que mesmo buscando na história os elementos a serem utilizados numa narrativa ficcional, o texto literário opera no campo de um “como se”. Iser afirma, no ensaio “O jogo do texto”, que “os autores jogam com os leitores”. Os ficcionistas usam diversas estratégias narrativas e efeitos estéticos a fim de dar corpo ao passado destacando aquilo que pretendem colocar em evidência. No texto, ele segue dizendo:
Ora, como o texto é ficcional, automaticamente invoca a convenção de um contrato entre autor e leitor, indicador de que o mundo textual há de ser concebido, não como realidade, mas como se fosse realidade. Assim, o que quer que seja repetido no texto não visa a denotar o mundo mas apenas um mundo encenado. Este pode repetir uma realidade identificável, mas contém uma diferença decisiva: o que sucede dentro dele não tem as consequências inerentes ao mundo real referido. Assim, ao se expor a si mesma a ficcionalidade, assinala que tudo é tão-só de ser considerado como se fosse o que parece ser; noutras palavras, ser tomado como jogo (2011, p. 107).
Esse contrato estabelecido entre autor e leitor, o “como se” apresentado por Iser, operacionaliza e completa os sentidos do que é dito por Aristóteles na Poética e por José Saramago quando entrevistado: é no como poderia ter acontecido que mesmo as narrativas com temáticas históricas funcionam. No “como se” dos romances Anatomia dos Mártires, Deixem falar as pedras e O teu rosto será o último, as personagens que viveram no tempo de Salazar, que sofreram as consequências de um sistema opressivo, ganham uma possibilidade de uma vida extra. Este “como se” tenta recompensar aqueles que antes derramaram sangue para que as novas gerações tivessem hoje uma vida de liberdade onde é, inclusive, possível escrever sobre aqueles que foram mortos e invisibilizados por Salazar. 
Além disso, aliado ao conteúdo está a forma. A ficcionalização do discurso histórico apresenta elementos formais que contribuem para que o conteúdo seja verossímil. Nos textos examinados nesta tese, algumas estratégias utilizadas são a escrita fragmentária, uma dinâmica narrativa no romance que se aproxima do ensaio, a escrita diarística, o discurso polifônico, entre outros. São formas que objetivam dar ao conteúdo o corpo apropriado. Ao se falar, por exemplo, de um relato de guerra, a escrita fragmentária aproximar-se-á da narrativa do trauma. O discurso polifônico ajuda a exprimir as diversas vozes que falam de um personagem histórico, real, que com o passar do tempo se converte em um mito. O diário é um espaço de liberdade que opera como um contraponto à censura das ditaduras. Ao falar sobre os romances de José Saramago que trazem a história como temática, em especial, o História do Cerco de Lisboa, em seu livro A ficcionalização da História, Márcia Valéria Zamboni Gobbi diz:
Vimos que a leitura da História, via romance, é uma operação cujo movimento fundamental é o jogo ambíguo, a tensão entre um afirmar e um duvidar, uma incorporação da História que, simultaneamente, questiona sua verdade. Esse movimento manifesta-se, na construção da narrativa, por meio dos aspectos que vimos analisando: a (con)fusão temporal, a sobreposição de espaços, o rebaixamento dos personagens, o rompimento das fronteiras entre as vozes dos personagens e a do narrador (que é, enfim, o regente dessa (des)ordem) (2011, p. 83).

Os elementos colocados como constituintes dessas narrativas que abraçam os temas e personagens históricos criam efeitos específicos que apontam, mais uma vez, para a importante questão que é: mesmo tratando da história, são textos ficcionais. Ambos são discursos que, entretanto, distanciam-se ao sistematizar os acontecimentos. Ao trazer para o seu texto Poética do pós-modernismo a problemática das escritas da história e da literatura,  Linda Hutcheon afirma:
O que a escrita pós-moderna da história e da literatura nos ensinou é que a ficção e a história são discursos, que ambas constituem sistemas de significação pelos quais damos sentido ao passado (“aplicações da imaginação modeladora e organizadora”). Em outras palavras, o sentido e a forma não estão nos acontecimentos, mas nos sistemas que transformam esses “acontecimentos” passados em “fatos” históricos presentes. Isso não é um “desonesto refúgio para escapar à verdade”, mas um reconhecimento da função de produção de sentido dos construtos humanos (HUTCHEON, 1991, p. 122).

Nos três romances examinados nesta tese, a narrativa histórica assume papel fundamental e não apenas decorativo. Neles, a história serve de fato como ponto de partida e elemento crucial para que as narrativas façam sentido. Entretanto, como apontado por Todorov, “quando a História serve de ponto de partida para suas ficções, o poeta pode tomar liberdades em relação ao desenrolar exato dos fatos, mas é para revelar sua essência oculta: nisso reside a superioridade da poesia sobre a História, já diziam os Antigos” (TODOROV, 2017, p. 396). É o que acontece em Anatomia dos Mártires, Deixem falar as pedras e O teu rosto será o último. Embora apareçam elementos centrais da história portuguesa, tratando-se de um espaço de ficção, há personagens e acontecimentos que fazem parte da liberdade dada aos ficcionistas. Segundo Ana Paula Arnaut aponta em Post-modernismo no romance português contemporâneo, 
Acreditamos, no entanto, no pleno direito que a ficção tem de pôr em causa a História (principalmente, como veremos, um certo tipo de ficção histórica), duvidando quer dos seus métodos quer das suas opções para conferir maior relevo a uma ou outra figura, a um ou a outro evento, assim instaurando, ou pelo menos propondo, novos cenários do que poderia ter acontecido. [...] Acreditamos, ainda, no âmbito dos novos rumos por que parece nortear-se a historiografia contemporânea, que alguma ficção procede também à reconstituição de certos percursos de mentalidade e de certos modos de vida. Numa linha adjacente, certa ficção leva ainda a cabo o redimensionamento e a reabilitação de figuras anónimas a quem, apesar de tudo, ainda se não confere e reconhece, na materialidade gráfica do discurso histórico, a devida importância na formação do que hoje somos como país e como povo (ARNAUT, 2002, p. 306).

Como acontecerá em romances de autores como José Saramago, António Lobo Antunes e Lídia Jorge, citando-se apenas alguns dos mais destacados e já consagrados ficcionistas contemporâneos, as personagens criadas para as páginas dos romances de João Tordo, David Machado e João Ricardo Pedro, embora não tenham de fato existido, são figuras representativas da ‘geração que não viveu enquanto se ia vivendo’. Retratar essas personagens como indivíduos, mesmo que não sejam reais, com suas vidas cotidianas, seus amores e sofrimentos, suas vontades e pensamentos, ajuda a tirar uma ideia usual de coletivo. Coletivamente, indivíduos se tornam apenas mais um número e o perigo disto está na despersonalização. Segundo segue afirmando Arnaut:
A “pessoa real” pode não existir nas páginas da obra mas, segundo julgamos, o facto de a narrativa ser pautada e enquadrada, de modo sistemático, por um leque de acontecimentos cuja dimensão e importância se encontram atestadas nas Histórias oficiais torna-se pertinente, também, para estabelecer uma catalogação genológica idêntica (ARNAUT, 2002, p. 311).
A literatura que reescreve a história tenta imaginar o mundo como poderia ter sido. Mesmo nas personagens inventadas, a literatura, que como a Nova História, muda o foco para aqueles que sempre foram considerados pequenos, pessoas cujas vidas em nada alteraram a macro história, consegue dar corpo à memória, mesmo que uma memória estilhaçada, fragmentada, em migalhas. No “como se” destes textos, os indivíduos fictícios ganham rostos, vozes e vidas que querem ser vistos, ouvidos e narrados.

1.3 A pós-memória


Não por acaso deram os gregos como mãe às musas Mnemosine. A personificação da memória dá à luz às personificações das artes na Antiguidade Clássica. A presentificação da memória pelo viés das artes, portanto, é essencial. Também a memória como área de conhecimento tem sido estudada e teorizada há muito na história da humanidade. Frances Yates encontra essa origem na memória compreendida como arte em si. Remontando esta gênese, Yates diz, em A arte da memória:
O estudioso da história da arte clássica da memória deve sempre lembrar que essa arte pertencia à retórica, como uma técnica que permitia ao orador aprimorar sua memória, o que capacitava a tecer longos discursos de cor, com uma precisão impecável. E foi como parte da arte da retórica que a arte da memória viajou pela tradição européia, sem ter sido jamais esquecida – pelo menos até tempos recentes –,  e que os antigos, guias infalíveis de todas as atividades humanas, traçaram regras e preceitos para aprimorar a memória (YATES, 2007, p. 18).

No seu texto, Yates mostra como o desenvolvimento de técnicas de memorização era importante em uma época em que os armazenadores de memória que hoje se usa não estavam disponíveis. Por este motivo, mesmo que esse caráter técnico da memória se distancie do que será a teorização da mesma, foi um elemento fundamental para tal. Das técnicas de palácios e teatros da memória, chega-se às distinções entre memória-hábito e memória-lembrança propostas por Bergson em Matéria e Memória. Com essa proposição, Bergson separa uma memória que vem da repetição contínua da memória que se vincula à imaginação, como afirma Ricoeur, “à memória que repete, opõe-se à memória que imagina” (RICOEUR, 2007, p. 44). Aleida Assmann explicita as diferenciações entre esses tipos de memória ao dizer:
Enquanto a mnemotécnica e os procedimentos técnicos de armazenamento estão preocupados em garantir que o conteúdo da memória armazenada seja idêntico ao conteúdo que será resgatado depois, no caso da memória natural há uma desvinculação desses dois atos. Experiência e recordação nunca se deixam harmonizar em conformidade plena. Entre ambas há um hiato em que o conteúdo da memória será deslocado, esquecido, obstruído, repotencializado ou reconstruído. Quanto mais as metáforas da memória fazem jus a essa dinâmica imanente das recordações, tanto mais elas realçam a dimensão temporal como fator decisivo e tanto mais fazem da reconstrução dos conteúdos da memória o verdadeiro problema em questão (ASSMANN, 2011, p. 191).
Em seguida, Halbwachs, que estudou filosofia com Henri Bergson, traz para a teorização da memória uma nova hipótese: os indivíduos recordam-se coletivamente. Em seu livro A memória coletiva, o sociólogo francês afirma:
Não basta reconstituir pedaço a pedaço a imagem de um acontecimento passado para obter uma lembrança. É preciso que esta reconstrução funcione a partir de dados ou de noções comuns que estejam em nosso espírito e também no dos outros, porque elas estão sempre passando destes para aquele e vice-versa, o que será possível somente se tiverem feito parte e continuarem  fazendo parte de uma mesma sociedade, de um mesmo grupo. Somente assim podemos compreender que uma lembrança seja ao mesmo tempo reconhecida e reconstruída (2003, p. 39).
As noções de que as lembranças se constroem coletivamente e de que a lembrança é, ao mesmo tempo, “reconhecida e reconstruída” é importante para a compreensão de outra teoria ligada à memória e que pautará esta tese. Embora as representações geracionais nos romances Anatomia dos Mártires, Deixem falar as pedras e O teu rosto será o último sejam diversificadas, os três ficcionistas que os escrevem pertencem à geração nascida imediatamente após a Revolução dos Cravos. Não tendo vivenciado esse período da história portuguesa, eles estão ainda suficientemente próximos para que conheçam o testemunho da geração (gerações) que o viveu (viveram) e ouçam os relatos diretamente dos pais e avós. Conforme observar-se-á nos capítulos sequentes, esses três romancistas voltam seus olhares para Portugal e narram acerca do período que essas gerações tão próximas ainda viveram.
O olhar para um passado recente, no entanto, não é um privilégio português. No seu livro The generation of postmemory: writing and visual culture after the holocaust, Marianne Hirsch aponta, na introdução, para uma nova conceituação acerca da memória:
“Pós-memória” descreve a relação que a “geração seguinte” mantém com o trauma pessoal, coletivo e cultural daqueles que vieram antes - para experiências que eles “lembram” apenas por meio de histórias, imagens e comportamentos entre os quais cresceram Mas essas experiências foram transmitidas a eles tão profunda e afetivamente que parecem constituir memórias por direito próprio. A conexão da pós-memória com o passado é, portanto, na verdade mediada não por recordação, mas por imaginação, investimento, projeção e criação. [...] Esses eventos aconteceram no passado, mas seus efeitos continuam no presente. Essa é, creio eu, a estrutura da pós-memória e o processo de sua geração (2012, p. 5)9.

O termo pós-memória é usado por Hirsch para descrever a forma como a geração que não vivenciou o trauma (que, conforme apresenta, pode ser pessoal, coletivo ou/e cultural) e, por extensão, a memória, relaciona-se com a geração que passou por essas experiências. A impressão causada por essas memórias transmitidas como uma herança familiar é tão expressiva que se converte em expressões artísticas. Marianne Hirsch traz como exemplificação o caso de Maus, graphic novel de Art Spiegelman, um filho de sobreviventes do Holocausto. Nele, Spiegelman narra a partir das experiências que não vivenciou, porém fazem parte da sua memória pela via do testemunho do pai. Através de narrativas como essa é possível que se perceba que:
descendentes de vítimas que sobreviveram, bem como de perpetradores e de espectadores que testemunharam eventos traumáticos massivos se conectam tão profundamente às lembranças do passado da geração anterior que identificam essa conexão como uma forma de memória, e que, em certas circunstâncias extremas, a memória pode ser transferida para aqueles que não estavam realmente lá para viver um evento10 (HIRSCH, 2012, p. 3).

No seu texto, Marianne Hirsch chama essa profunda conexão com as memórias da geração anterior de “atos de transferência", um termo que é empregado por Paul Connerton. E embora memória e pós-memória não sejam a mesma coisa, a força afetiva desta produz efeitos mentais semelhantes àquela (cf. HIRSCH, 2012, p. 31). O indivíduo, mesmo não tendo passado pelo trauma, pela dor da memória herdada, consegue sofrer e senti-la profundamente como parte de si.
O elo entre a geração que herda com a geração que vivenciou é tão forte que a narrativa da memória (e quando fala-se em narrativa, não é apenas ao texto escrito que refere-se, mas à toda manifestação discursiva, sobretudo as ligadas às artes.) torna-se uma necessidade. A literatura, assim como acontecerá com a fotografia, as artes plásticas e o testemunho, funcionam como uma espécie de “conexão viva” entre o passado e os indivíduos que o recordam (cf. HIRSCH, 2012, p. 33). Ao trazer para as páginas de livros, para telas e exposições as recordações da memória da geração passada, aqueles que herdam compartilham ainda com outros o peso do tempo que carregam. Em Espaços da recordação, Aleida Assmann diz:
Chama a atenção o fato de que a arte começa a se ocupar mais fortemente da memória justamente no momento em que a sociedade faz pressão para que a memória se perca ou seja apagada. Nesse contexto, a memória artística não funciona como armazenador, mas estimula os armazenadores, ao tematizar os processos de lembrar e esquecer. Pois para os artistas não se trata de usar armazenadores tecnológicos; eles buscam, sim, um “glossário de sentimentos”, em que reconhecem uma fonte de elementos artísticos (ASSMANN, 2011, p. 26).

A arte produzida a partir da pós-memória, portanto, sob o ponto de vista apresentado por Assmann, estaria menos preocupada em guardar uma espécie de registro dos acontecimentos e, mais voltada para lembrar que é preciso se lembrar. A arte traz a crise da memória como tema. Ainda tomando as constatações de Aleida Assmann como arcabouço, lê-se que:
A nova arte sobre a memória age em outra área. Ela não precede, mas sim sucede o esquecimento, pois não é uma técnica ou medida preventiva. Ela é, no melhor caso, uma terapia para traumas, uma coleção cuidadosa de restos espalhados, um balanço das perdas. [...] Os artistas que trabalham sobre a memória, no começo deste novo milênio, encontram-se em uma situação diferente. Eles chegaram à cena da catástrofe depois que ela aconteceu, e não se pode mais pensar numa arte que pudesse estabelecer uma ponte de memória entre o agora e o então. Para eles não há mais nada a reconstruir ou mesmo reconstituir: deve-se tão somente recolher os restos, salvaguardar, ordenar e conservar os vestígios de que ainda sobrou de relíquias espalhadas. Esses artistas que trabalham com a memória não documentam, com seu trabalho, os grandes feitos da lembrança que tratam da morte, mas fazem o balanço da perda (ASSMANN, 2011, p. 386).

Esta é a característica em que a arte produzida na pós-memória mais se distingue: nas narrativas de autores como os já exaustivamente citados José Saramago, António Lobo Antunes e Lídia Jorge, as feridas do salazarismo, as memórias de um tempo opressivo em que se vivia sob uma ditadura já foram expostas. Na produção da novíssima geração, entretanto, a diferença se fixa no fato de que não se tenta mais documentar o passado, mas ‘fazer um balanço da perda’11.
A manifestação desse relacionamento inter-geracional se conecta à carga histórica do que foi o século XX. O período agitado que vai, aproximadamente, de 1914, com a Primeira Guerra Mundial, à 1988, com a derrubada do muro de Berlim, trouxe imagens de catástrofes que estão ainda muito frescas nas mentes. Embora as guerras e turbulências históricas estejam presentes desde o início da história da humanidade, nunca antes teve-se tantos meios de manter tudo registrado, fixo na memória histórica. Em Mutações da literatura do século XXI, Leyla Perrone-Moisés afirma que:
Jamais o homem carregou uma memória histórica tão vasta quanto a atual. E a memória recente, a do século XX com suas guerras e horrores, é culpabilizante. Daí a frequência do tema do espectro na literatura contemporânea. O espectro é o morto mal enterrado, que volta para cobrar alguma coisa mantida em instância. Por outras palavras, é o passado que se recusa a morrer (PERRONE-MOISÉS, p. 150).

Apesar do quão brutal possa parecer a reflexão e recuperação dos traumas e feridas causadas pelo século XX, há nessas narrativas um trabalho fundamental: em Diante do extremo, Todorov diz que:
Do ponto de vista não mais de si, mas da humanidade (que cada um pode, por sua vez, emprestar), uma vida não é vivida em vão se dela restar um rastro, uma narrativa que se junta às inúmeras histórias que constituem nossa identidade, contribuindo assim, ainda que em ínfima, para tornar esse mundo mais harmonioso e mais perfeito. Este é o paradoxo dessa situação: as narrativas do mal podem produzir o bem ( 2017, p. 145).

Como anteriormente citado referenciando-se a Marcuse, recorre-se ao passado quando no presente falta algo. Ao falar sobre os textos com elementos históricos de Shakespeare, Assmann recorda que essas peças ‘não realçam na história o que passou, mas o que é presente’ (cf. 2011, p. 87). O texto literário, quando registra a história passada, na verdade está falando sobre o tempo presente em que ele é escrito. O ato de recordar, mesmo a pós-memória, esta memória que é herdada da geração que experienciou o trauma, é um trabalho constante de ressignificação. Leyla Perrone-Moisés diz que “a presença do passado nas obras atuais não se manifesta de modo diacrônico, como nos manuais de história literária, mas de modo sincrônico, que é o modo da memória” (2016, p. 116-7). A memória traz consigo também, as marcas do momento em que é recordada. Lembra-se porque é uma necessidade lembrar, porque, ao se fazer o balanço da perda é possível que se compreenda o que pode ser feito das ruínas deixadas. Como apontado por Ronaldo Lima Lins em A indiferença pós-moderna, “o passado não guarda apenas revelações esquecidas; guarda valores, muitos deles úteis, se fosse possível trazê-los à modernidade para esquentar, com um cobertor de fraternidade, a frieza das nossas relações” (2006, p. 20).

1.4  A herança: o novíssimo contemporâneo


Gonçalo M. Tavares, Afonso Cruz, Patrícia Portela, João Tordo, Joana Bértholo, Nuno Camarneiro, Valter Hugo Mãe e José Luís Peixoto são alguns dos mais recentes nomes da literatura portuguesa que têm se espalhado pela Europa e mundo afora, sendo traduzidos, lidos e estudados. Há também alguns nomes mais tímidos como Tiago Patrício, Bruno Margo, Rui Cardoso Martins, Sandro William Junqueira cujos textos, mesmo que ainda não tão lidos ou traduzidos, conquistam o olhar crítico devido a sua qualidade literária.
Conceber a ideia de um romance genuinamente português faz cada vez menos sentido ao se pensar na pluralidade cultural impulsionada pela globalização do século XXI. Do mesmo modo, pensar numa unidade na escrita desses escritores não parece adequado. Nos diversos autores citados há uma variedade de temas, escolhas narrativas, perspectivas. Encontram-se romances distópicos, policiais, históricos, entre outros, assim como observa-se na geração anterior.
Ao longo deste texto repete-se algumas vezes a expressão geração pós-2000. Quando fala-se em geração na literatura, pode-se estar a referir a um “grupo de escritores contemporâneos e coetâneos que comungam dos mesmos ideais, respondem aos mesmos desafios históricos, partilham a mesma estética e que muitas vezes procuram construir uma obra com características comuns” (CEIA, Carlos, 2009)12. A supracitada geração pós-2000, entretanto, não deve ser tomada como um grupo organizado de escritores a partilhar ideais. Utiliza-se a palavra geração, nomeadamente, a dita geração dos Novíssimos, para contemplar um grupo de escritores que começa a escrever em Portugal no período posterior aos anos 2000. Embora compartilhem um tempo e espaço no cenário literário, dificilmente pode-se conectá-los tomando suas produções literárias como argumento, apesar de, eventualmente, encontrar-se pontos de contatos entre as escritas de alguns autores. 
Falar sobre a literatura produzida pelos novíssimos coloca em pauta uma problemática relacionada às definições. Esses autores recentes têm ainda muito pouco sobre si escrito e, as conclusões ainda estão em suspenso. Trazer para a análise romances produzidos no século XXI e, escritos por autores que começam a arte de narrar nesse novo século, traz para o foco algumas questões importantes. Há um diálogo contínuo sendo estabelecido entre o que se escreve e o que já foi escrito. Ao mesmo tempo, há uma tentativa da crítica em determinar as marcas de uma ruptura com o pregresso.
Em O romance português contemporâneo, Miguel Real tenta sistematizar a literatura produzida pela geração dos novíssimos estabelecendo uma linha temporal que parte da produção literária de 1950. Usando uma árvore como metáfora, Real apresenta o fim do realismo queirosiano na década de 50 como as raízes dessa estrutura que seria o romance português contemporâneo. O processo de desconstrução do estilo realista é marcado pela publicação de três obras nos anos de 1953 e 1954: Uma Abelha na Chuva, de Carlos de Oliveira, O Trigo e o Joio, de Fernando Namora e Sibila, de Agustina Bessa-Luís. (cf. REAL, 2012, p. 71). Muito tempo antes, Eduardo Lourenço indica, em “Uma literatura desenvolta ou os filhos de Álvaro de Campos”, o período entre 1953 e 1963, especificamente da publicação de Sibila à Rumor Branco, como uma época não somente com um “número considerável de obras particularmente brilhantes, fenómeno já de si singularíssimo, mas obras de um tom e de uma estrutura afins, cujo segredo é natural buscar-se na conformidade da sua expressão ao espírito de época que as viu nascer” (p. 924).
A época da qual fala Lourenço é o período do salazarismo. Na expressão literária, o conflituoso período histórico português foi representado, sobretudo, pela narrativa neorrealista. Recorrendo-se mais uma vez à Eduardo Lourenço, dessa vez, ao ensaio “Psicanálise mítica do destino português” presente em O labirinto da saudade, afirma-se que:
Tal foi o papel histórico considerável do movimento “neo-realista”, cuja história cultural e ideológica, na sua complexidade, está por fazer, mas sem o qual a nossa futura e actual relação de portugueses com Portugal é simplesmente incompreensível. É sob o seu império ou na sua movência que se cria em relação à clássica imagem de Portugal como país cristão, harmonioso, paternal e salazarista, suave, guarda-avançada da civilização ocidental antimarxista, uma outra-imagem que não é exactamente uma contra-imagem, mas uma complexa distorção desse protótipo que nalguns aspectos se apresenta como o pólo oposto dela (sobretudo pela “ocultação” do carácter repressivo de índole cristã). [...] É o carácter obscurantista, a prepotência de classe ou a glosa romanesca da multiforme miséria do povo português que servem de alvo ou justificam uma lenta mas implacável erosão do espírito burguês provincial do salazarismo, sem aliás lhe alterar nem a boa consciência cultural nem política (2016, p. 40-1). 

Nas constatações indicadas por Eduardo Lourenço nos dois ensaios, a literatura produzida em Portugal se apresenta como um meio de expor uma imagem diferente da sociedade portuguesa daquela propagada pelo salazarismo. Uma literatura engajada, mas não tendenciosa (cf. ADORNO, 1991, p. 54), o que refletirá nas produções dos anos seguintes. Se pode-se pensar nas produções artísticas dessa década como as raízes, os anos 60 e 70 são apresentados como o tronco, onde romances como Paixão, de Almeida Faria, Manual de Pintura e Caligrafia, de José Saramago e Os cus de Judas, de António Lobo Antunes desconstroem as estruturas da narrativa clássica. Ana Paula Arnaut aponta para O Delfim como o romance que se emancipa do neorrealismo:
É certo que anteriormente a Maio de 1968, data a todos os títulos emblemática também por ser  a da publicação de O Delfim, é possível corroborar a existência de obras que, de modo mais ou menos flagrante, desagregam e, consequentemente, se afastam  do Neo-Realismo, em primeiro plano desde a publicação de Gaibéus em 1939. Este afastamento consubstancia-se, em alguns casos, mais em termos formais e estruturais do que propriamente temáticos e, em outros, pelos novos vínculos existencialistas evidenciados (2012, p. 80).

Com vozes já consagradas e novos nomes apontando para alguns dos herdeiros da Literatura Portuguesa, os anos 80 e 90, os ramos e folhas dessa árvore literária, trouxeram narrativas como Levantado do Chão e Memorial do Convento, de José Saramago, cuja temática histórica é apresentada de forma desconstruída e Instrução dos amantes e Nas tuas mãos, de Inês Pedrosa, romances que, segundo coloca Erivelto da Silva Reis em sua tese A estética da intimidade em Inês Pedrosa, a “estética predominante é a da revelação da intimidade das personagens" (REIS, 2018, p. 54). João Barrento apontará, em A chama e as cinzas, a literatura produzida em Portugal no quarto final do século XX como tendo o olhar voltado para trás, para a distância e para dentro. A literatura cujo olhar se volta para trás se refere aos romances que trazem a História como tema, as histórias mais antigas, as memórias da ditadura salazarista e da Revolução dos Cravos. A literatura que olha para a distância, toma territórios da África, América e até da Ásia, assim como suas respectivas culturas, como tema. Já o olhar voltado para dentro representa uma literatura que se volta para si mesmo e para o ato da escrita. (cf. 2016, p. 18-9). Estes distintos olhares apresentar-se-ão tanto na literatura produzida por autores que narram desde o período do salazarismo, quanto na daqueles que começam a escrever nas supracitadas décadas. É de interesse se ter em mente que há uma coexistência intergeracional: quando na literatura surgem novos autores, os que já escrevem não deixam de existir. Assim sendo, quando se fala nos diálogos estabelecidos entre as diferentes gerações, muitas vezes eles são recíprocos.
No oitavo número da Revista de Estudos Literários, a problemática das definições de termos adequados para a literatura que se faz agora aparece como temática sob o título “Do post-modernismo ao hipercontemporâneo: os caminhos das literaturas em língua portuguesa”. Na nota prévia do texto, Carlos Reis aponta para os problemas gerados por uma tentativa de delimitação de termos como pós-moderno e contemporâneo e o agora empregado termo hipercontemporâneo. A questão levantada começa pela dificuldade em se classificar como contemporâneos autores temporalmente tão distantes quanto Jorge de Sena de Gonçalo M. Tavares (cf. 2018, p. 7-10).
O processo que envolve a determinação de delimitações periodológicas na literatura não ocorre sem que haja pontos de vista distintos, particularmente quando há pouco distanciamento temporal.  Por exemplo, embora admita a adoção do termo pós-moderno para falar sobre o período histórico, Leyla Perrone-Moisés pontua que é diante da aplicação à estética e, por extensão, à literatura, que o termo se torna impreciso. Segundo a teórica:
as peculiaridades apontadas pelos teóricos como “pós-modernas” são pouco convincentes. Ao termo de três décadas de tentativas de definir literatura pós-moderna, acumularam-se as imprecisões e as simplificações. Linda Hutcheon, que está entre os principais teóricos da literatura pós-moderna, chegou à conclusão de que se tratava mais de uma “problemática” do que de uma “poética” (2016, p. 41).
Perrone-Moisés chama ainda atenção para o que chama “atitude consumista” da literatura pós-moderna ao se alimentar da modernidade. Na falta de um termo que considere mais adequado, utiliza literatura contemporânea mas, afirma que o termo apresentará problemas futuros quando o tempo mudar o sentido do adjetivo empregado (cf. PERRONE-MOISÉS, p. 45).
No ensaio “O que é o contemporâneo?”, em seu livro homônimo, Giorgio Agamben começa por perguntar “do que e de quem somos contemporâneos”. Ao passar para as definições, ele afirma que ‘a contemporaneidade é uma singular relação com o próprio tempo tomando dele distância’ (cf. 2009, p. 59) e que “contemporâneo é aquele que mantém fixo o olhar no seu tempo, para nele perceber não as luzes, mas o escuro” (IBIDEM, p. 62). Sob tal ponto de vista, a literatura do contemporâneo aparece como uma literatura que, conceitualmente, afastava-se do seu tempo e sob tal perspectiva, enxerga as sombras e é sobre elas que narra. Narra sobre um tempo inacabado, o que combina com o inacabamento do romance. Em Teoria do romance II: as formas do tempo e do cronotopo, Bakhtin afirma:
É constitutivo de todo romance (do gênero romanesco por sua natureza) o contato com a realidade inacabada. Não se trata do mundo pátrio nem do mundo estranho, mas do mundo em que nós também vivemos, no qual também nós poderíamos vivenciar todas essas aventuras, e todas as pessoas, assim como nós, são pessoas privadas, não são heróis épicos inacessíveis a nós [...]. Aí já se esboça o contato com a realidade inacabada (do autor e do leitor). Esse mundo é aberto, não é concluído ou fechado como uma epopeia. Só é fechado formalmente em termos fabulares, só para certos heróis, mas o próprio mundo permanece o mesmo de antes. E esse tempo aventuresco está inserido no nosso tempo (e não distanciado dele, como na epopeia) (BAKHTIN, 2018, p. 242).
O texto introdutório do citado oitavo número da Revista de Estudos Literários assinado por Ana Paula Arnaut e Ana Maria Binet, aponta para algumas das características que serão usadas como ponto de referência para se atribuir o termo hipercontemporâneo à escrita produzida após 2000. As questões ambientais, “um intimismo que parece ser um voltar as costas a um mundo que é só dispersão e ausência de sentido”, a busca pelas raízes, o mundo virtual como realidade alternativa, e um processo de desumanização são alguns dos elementos presentes nestas narrativas (cf. 2018, p. 12). Segundo apontam as autoras: 
Fruto da globalização, das novas tecnologias, essa literatura que marca os nossos panoramas literários, seja no continente europeu, seja no americano ou no africano, é um reflexo de um mundo em profunda mudança, onde as mentes e os corpos se expõem ao domínio da ciência e da tecnologia, integrando-as no seu foro interno. Assim, a literatura hipercontemporânea põe em cena personagens híbridos, homens-máquina, máquinas antropomórficas, oferecendo-nos uma visão do futuro que nos atemoriza. A violência político-religiosa, que marca profundamente as nossas sociedades, especialmente desde o 11 de Setembro de 2001, percorre uma literatura onde o medo da morte, que tínhamos conseguido eufemizar, volta brutalmente, através da consciência de que esta se pode sobrepor às estruturas socioculturais, que tinham como objetivo mantê-la à distância, e se revelam impotentes perante a força do tsunami que nos assola, particularmente na Europa (2018, p. 11).

A questão que se coloca mais uma vez, partindo do ponto colocado pelas autoras Sob um ponto de vista literário, talvez seja ainda cedo demais para se estabelecer delimitações definitivas entre contemporâneo e hipercontemporâneo. Por outro lado, sob a perspectiva da historiográfica, talvez seja essa uma necessidade. Em um tempo que torna-se cada vez mais acelerado, a contemporaneidade ganha, cada vez mais, o caráter de hiper. De todo modo, recorre-se novamente à Perrone-Moisés para pensar acerca dessa questão. A literatura pertence a quem a faz e serão os escritores os responsáveis por mostrar qual será o direcionamento desta. Segundo afirma:
A importância da literatura na cultura contemporânea não pode ser defendida fora de uma prática. São os escritores e não os teóricos que definem, em suas obras, as mutações da literatura. Os valores acima sintetizados foram definidos pela modernidade, mas alguns dos valores modernos são atualmente menosprezados pelos escritores. A busca do “novo”, por exemplo. O “make it new” das vanguardas não é mais um mandamento. A originalidade ainda é um valor, porque o gosto pela informação nova é atemporal. Mas a maioria dos romancistas atuais não busca mais, como Joyce ou Guimarães Rosa, uma transformação inovadora da língua ou da técnica narrativa. De modo geral, o romancista contemporâneo continua usando técnicas narrativas tradicionais, apenas sutilmente renovadas com respeito aos diálogos e às descrições. [...] Os valores buscados numa narrativa ou num poema, atualmente, são a veracidade, a força expressiva e a comunicativa (PERRONE-MOISÉS, p. 35-6).

A geração que escreve após os anos 2000 é herdeira de um período de efervescência literária portuguesa. Saramago ganha o Prêmio Nobel da Literatura em 1998 e este fator certamente contribui para que outros autores de portugueses ganhem notoriedade pelo mundo afora13. Essas novas vozes trazem narrativas que revelam o legado que lhes foi transmitido e em movimentos de aproximação e distanciamento tecem uma nova literatura portuguesa. Em Mutações da literatura do século XXI, Leyla Perrone-Moisés aponta para a relação que se estabelece com o que é antecedente: 
Como acontece com todos os herdeiros, muitos deles dilapidam a herança, trocando em miúdos, produzindo uma infinidade de pequenas obras de mero entretenimento, ou nem isso. Outros a gastam moderadamente, seguindo os ensinamentos de seus pais e avós. Mas alguns sentem mais intensamente o peso da herança e procuram ser dignos daqueles que a legaram. Certos escritores atuais têm feito o luto dos antepassados em obras metaliterárias, que os citam e celebram seus feitos (PERRONE-MOISÉS, p. 50).

Embora não se queira desconsiderar os séculos de literatura produzida antes das décadas mencionadas neste subcapítulo, séculos que obviamente pautaram a escrita que se delineou aproximadamente a partir dos anos 50, a herança literária aqui referenciada é a recebida diretamente dos pais e avós literários. Os escritores da geração pós-2000 entram no cenário português depois de importantes movimentações literárias e históricas. Os seus mortos ainda não estão descansando, o trabalho do luto ainda está em curso. Sobre esses espíritos que continuam a rondar, Derrida diz em Espectros de Marx que “é preciso contar com eles. Não se pode não dever, não se pode não poder contar com eles” (DERRIDA, 1994, p. 13). Em uma entrevista publicada em De que amanhã…, Derrida diz ainda:
é preciso primeiro saber e saber reafirmar o que vem “antes de nós”, e que portanto recebemos antes mesmo de escolhê-lo, e nos comportar sob esse aspecto como sujeito livre. Ora, é preciso (e este é preciso está inscrito diretamente na herança recebida), é preciso fazer de tudo para se apropriar de um passado que sabemos no fundo permanecer inapropriável, quer se trate aliás de memória filosófica, da precedência de uma língua, de uma cultura ou da filiação em geral. Reafirmar, o que significa isso? Não apenas aceitar essa herança, mas relançá-la de outra maneira e mantê-la viva. Não escolhê-la (pois o que caracteriza a herança é primeiramente que não é escolhida, sendo ela que nos elege violentamente), mas escolher preservá-la viva. A vida, no fundo, o ser-em-vida, isso talvez se defina por essa tensão interna da herança, por essa reinterpretação do dado do dom, até mesmo da filiação. Essa reafirmação, que ao mesmo tempo continua e interrompe, no mínimo se assemelha a uma eleição, a uma seleção, a uma decisão (DERRIDA, 2004, p. 12-3).

A geração dos Novíssimos, portanto, mesmo em face de experiências narrativas, novas possibilidades estéticas, técnicas e conteúdos inovadores, não pode negar a sua herança. O equilíbrio sempre buscado entre tradição e inovação dentro da estrutura do romance é mantido como o ideal. Muitos dos elementos experimentados pela geração anterior são repetidos pela geração pós-2000. Perrone-Moisés aponta para algumas características apresentadas como pós-modernas como a intertextualidade, a paródia, a metalinguagem, o ludismo e a ironia e mostra como esses elementos são já usados em textos hoje clássicos (cf. 2016, p. 42-3). O que se contrasta entre as duas gerações é uma intensificação dessas experimentações. O desconcerto gerado por uma escrita fragmentária não é privilégio dos autores que escrevem após 2000. O discurso polifônico, a reescrita da história, a hibridização do gênero, por exemplo, não são novidade, mas são mais explorados pelos novíssimos. Conforme apontado por Derrida, os herdeiros, no presente caso, os herdeiros da literatura portuguesa, ‘relançam sua herança de uma outra maneira e a mantém viva’. Não escolhem o que herdam, mas escolhem manter essa herança viva. 


%\newpage


\end{document}