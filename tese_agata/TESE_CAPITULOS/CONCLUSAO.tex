\documentclass[../DISSERTACAO_MAIN.tex]{subfiles}

	Neste trabalho, que foi publicado pela revista PeerJ em Janeiro de 2019 \cite{Vieira2019}, montamos e anotamos os primeiros 14 mitogenomas para a subfamília Pseudomyrmecinae. Para tal, utilizamos uma metodologia que se baseia no uso de dados públicos de diferentes fontes e tipos, em conjunto com a aplicação de softwares gratuitos de bioinformática para a manipulação desses dados. As sequências obtidas foram utilizadas para estudar a sintenia, genômica comparativa e relações filogenéticas desses organismos, fornecendo informações valiosas sobre a filogenia e evolução de Pseudomyrmecinae, como: 
	
	\begin{itemize}
		\item [(i)] identificação de quatro regiões putativas de inserção nucleotídica em mitogenomas do gênero Pseudomyrmex;
		\item [(ii)] corroboração de que as associações mutualísticas com plantas encontradas na subfamília são parafiléticas, tendo ocorrido independetemente múltiplas vezes no clado;
		\item [(iii)] indicação de que o grupo de espécies P. ferrugineus é monofilético, enquanto o grupo P. viidus é parafilético; e
		\item [(iv)] corroboração da monofilia dos gêneros Pseudomyrmex e Tetraponera.
	\end{itemize}

	Dados mitocondriais em outros clados de formigas, mesmo que limitados, nos permitiram ampliar nosso escopo e estudar a família Formicidae como um todo. Isso nos possibilitou elucidar relações de grupo irmão para a família, como a descrita entre Pseudomyrmecinae e Dolichoderinae, assim como o caráter monofilético de todas as subfamílias analisadas. Entretanto, uma definição mais precisa das relações entre os diferentes grupos de formigas idealmente devem se valer de grandes datasets genômicos e concatâmeros de centenas a milhares de genes, atualmente indisponíveis. Os baixos valores de bootstrap observados em alguns nós indicam que os dados mitocondriais disponíveis no momento não apresentam variabilidade o bastante para elucidar algumas relações, muito embora isso possa mudar com a ampliação tanto da quantidade quanto da cobertura taxonômica de genomas mitocondriais. As sequências mitocondriais montadas abarcam uma porção considerável da biodiversidade de Pseudomyrmecinae e serão úteis em novos estudos sobre a evolução e conservação desse grupo.
	
	Este trabalho praticamente dobra o número de mitogenomas de formiga completos disponíveis sem custos adicionais de sequenciamento. Uma vez que há poucos grupos de formigas com genomas mitocondriais completos disponíveis, o aprimoramento da cobertura mitogenômica é necessário para uma melhor resolução e robustez de filogenias em larga escala para o clado. A metodologia apresentada aqui também pode ser usada para estudar os já citados clados de monofilia duvidosa da subfamília Myrmicinae (Brady et al., 2006; Ward, 2011; Ward et al., 2015) ou grupos notoriamente parafiléticos, como o gênero Camponotus da subfamília Formicinae (Blaimer et al., 2015). Com base nesses resultados, enfatizamos que a cobertura filogenética cada vez maior dos bancos de dados públicos, associada à presença de sequências mitocondriais em diferentes tipos de dados de sequenciamento, torna a mitogenômica no-budget a abordagem ideal para o estudo da diversidade de espécies. Trabalhos que utilizem dados públicos para a montagem e análise de mitogenomas, como este e o projeto em andamento “100 MITO”, são possivelmente o caminho mais rápido para se obter amplas árvores filogenéticas que representem a história evolutiva das espécies da forma mais fidedigna possível.