\documentclass{article}
\usepackage[utf8]{inputenc}
\usepackage[version=4]{mhchem}

\title{Fórmulas}
\author{gabrieldeusdeth}
\date{July 2020}

\begin{document}
\maketitle

Dissociação de um ácido:

\begin{equation}
    \Huge\ce{ AH + H2O <=> H3O^+ + A^- }    
\end{equation}

Fórmula Ka:

\begin{equation}
    \Huge K_a = \frac{[H_3O^+] [A^-]}{[HA]}
\end{equation}

Fórmula pKa:

\begin{equation}
    \Huge pK_a = -logK_a
\end{equation}

Fórmula pH:

\begin{equation}
    \Huge pH = -log[H^+]
\end{equation}

Equação de Henderson-Hasselbalch: Relação entre pH e pKa 

\begin{equation}
    \Huge pH = pK_a + log\frac{[A^-]}{HA}
\end{equation}

Reescrevendo Henderson-Haselbalch, temos:

\begin{equation}
    \Huge 10^{pH-pK_a} = \frac{[A^-]}{[HA]}
\end{equation}

\end{document}
