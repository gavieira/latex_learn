\documentclass[../DISSERTACAO_MAIN.tex]{subfiles}

\begin{document}

\section{Aquisição de dados}

	Quatorze datasets paired-end obtidos por sequenciamento Illumina foram baixados do EMBL Nucleotide Archive (\url{https://www.ebi.ac.uk/ena}) no formato de arquivo SRA (consulte a Tabela 1). Esses conjuntos de dados contêm tanto reads mitocondriais quanto nucleares que foram convertidas para FASTQ usando o software fastq-dump (com parâmetros –readids e –split-files) que integra o pacote SRAtoolkit.2.8.2. 

\section{Montagem e anotação do genoma mitocondrial}
	
	Cada dataset completo foi usados como entrada para a montagem de novo usando NOVOPlasty2.6.3 (Dierckxsens et al., 2016) com os valores padrão dos parâmetros. Já que NOVOPlasty foi nosso montador principal e \citeauthoronline{Dierckxsens2016} (\citeyear{Dierckxsens2016}) recomendam o uso de dados não trimados nesse software, optamos por usar os datasets brutos como entrada para a montagem. A única exceção foi o dataset de Tetraponera rufonigra, que teve de ser ajustado com o software Trimmomatic v.0.36 (Bolger, Lohse \& Usadel, 2014) para produzir sequências com o mesmo comprimento em pares de base. Este controle dos dados foi realizado ao ajustar o parâmetro MINLEN do Trimmomatic para 125, que é o comprimento das maiores reads encontradas no dataset. Com isso, descartamos sequências menores e mantemos apenas aquelas de tamanho máximo para serem usadas como entrada na montagem inicial. Montagens NOVOPlasty precisam de uma sequência (denominada seed ou semente) que é utilizada para identificar uma read mitocondrial do dataset, a qual por sua vez será usada para iniciar a montagem (Dierckxsens et al., 2016). As seeds foram selecionadas utilizando sequências de COX1 (Citocromo Oxidase I) da mesma espécie (quando disponíveis) ou utilizando regiões de COX1 de espécies proximamente relacionadas. As montagens de mitogenoma preliminares realizadas pelo NOVOPlasty foram utilizadas como referência para uma segunda etapa de montagem utilizando o software MIRA v.4.0.2 com parâmetros padrão (Chevreux et al., 1999). NOVOPlasty não gera um arquivo de alinhamento mostrando as sequências mapeadas à montagem, então MIRA foi utilizado para mapear reads brutas  à montagem preliminar e permitir a análise de cobertura da sequência mitocondrial consenso nos próximos passos. Quando a primeira montagem não gerou o mitogenoma completo, nós usamos o MITObim v.1.9 (Hahn et al., 2013) sem alterar seus parâmetros. Este programa realiza montagens MIRA sucessivas para estender o(s) contig(s) mitocondrial(ais) e fechar pequenas lacunas da montagem, gerando a versão final e circularizada do genoma mitocondrial.	
	
	O software Tablet versão 1.17.08.17 (Milne et al., 2012) foi usado com valores paramétricos padrão para verificar a cobertura de reads e a circularização dos mitogenomas completos. O processo de anotação automática foi realizado usando MITOSWebServer (Bernt et al., 2013) sem a alteração de parâmetros. Em seguida foi realizada uma etapa de curadoria manual com o software Artemis v.17.0.1 (Carver et al., 2012) usando a tabela de código genético número cinco (correspondente à mitocôndria dos invertebrados) para identificar os limites das fases abertas de leitura (Open Reading Frames ou ORF’s). 
	
	Já que o genoma mitocondrial dá origem a um grande mRNA policistrônico que então é clivado (Boore, 1999), sobreposições gênicas poderiam incorrer na formação de proteínas, rRNAs ou tRNAs não funcionais. É então razoável pensarmos que mitogenomas nos quais as features não se sobrepõem são energeticamente mais econômicas para a célula e, consequentemente, foram selecionados ao longo da evolução. Partindo dessa premissa, tentamos ao máximo evitar sobreposições durante a anotação das sequências mitocondriais. Assim sendo, os limites gênicos dos tRNAs e rRNAs foram mantidos de acordo com os resultados do MITOS Web Server, salvo quando encontrada sobreposição entre duas features (gene codificador de proteína - PCG, tRNA ou rRNA) na mesma fita. Nesse caso, os nucleotídeos sobrepostos foram retirados de uma das features para que a sobreposição fosse completamente removida. O D-loop não foi explicitamente anotado, já que seu caráter hipervariável e de baixa complexidade (Moritz, 1994; Vanecek et al., 2004; Zhang \& Hewitt, 1997) torna difícil estabelecer limites precisos para essa região, em especial quando não se tem uma referência próxima. Entretanto, com base na análise comparativa de sintenia em Formicidae e no fato do D-loop ser geralmente a maior região intergênica do genoma mitocondrial (Liu et al., 2015; Zhang et al., 2016; Huang et al., 2017), identificamos que essa região variável se encontra entre o rrnS e trn-M.
	
	Para os PCGs, em vários casos se fez necessário expandir a anotação fornecida pelo MITOS Web Server de forma a englobar a maior ORF que não apresente sobreposição com outras features na mesma fita. Então, essa ORF foi utilizada como entrada na versão online do BLASTp (Altschul et al., 1997), sendo alinhada contra o banco de sequências pertencentes a família Formicidae do GenBank. As informações obtidas por esse alinhamento contra sequências mitocondriais de outras formigas nos permitiu considerar a conservação de sequências entre as espécies e determinar o tamanho mais provável da proteína, refinando a anotação. Seguindo esse procedimento, nós alcançamos uma decisão racional, com base em genômica comparativa, sobre os limites gênicos. O conteúdo de AT para (i) o genoma mitocondrial completo; e (ii) a região intergênica que contém o D-loop foram calculados usando o programa online OligoCalc (Kibbe, 2007) com os valores padrão para seus parâmetros.

\section{Análises filogenômicas}

	As relações filogenéticas de Formicidae foram reconstruídas usando (i) os 14 mitogenomas completos por nós produzidos juntamente com (ii) todos os outros 15 genomas mitocondriais completos atualmente disponíveis para o clado; e (iii) dois mitogenomas de abelhas (família Apidae) utilizados como grupos externos. Duas árvores filogenéticas foram construídas usando (i) toda a sequência mitocondrial e (ii) o conjunto de genes concatenados de todos os 13 genes codificadores de proteínas (PCGs). Para o primeiro, editamos manualmente as sequências para iniciar no gene COX1 quando necessário e alinhamos os mitogenomas inteiros usando o software ClustalW v.2.1 usando os parâmetros padrão (Thompson, Gibson \& Higgins, 2003). Para o segundo, alinhamos e concatenamos os nucleotídeos de todos os PCGs usando o programa Phylomito (\url{https://github.com/igorrcosta/phylomito}) sem alteração de seus parâmetros. Modeltest (Posada \& Crandall, 1998) foi executado através do software MEGA7 (Kumar, Stecher \& Tamura, 2016) para os dois conjuntos de dados e identificou o modelo GTR + G + I como o modelo de substituição de nucleotídeos que melhor explica a variação das sequências. As sequências alinhadas foram usadas como entrada para uma análise de Máxima Verossimilhança (Maximum Likelihood ou ML) usando o MEGA7. A reamostragem foi realizada por bootstrap usando 1000 réplicas. O software BRIG (Blast Ring Image Generator) v.0.95 (Alikhan et al. 2011) foi utilizado com valores paramétricos padrão para comparar e visualizar todos os mitogenomas de Pseudomyrmecinae produzidos aqui.

\end{document}