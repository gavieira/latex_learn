\documentclass[../DISSERTACAO_MAIN.tex]{subfiles}

Usando um protocolo ligeiramente modificado para a realização de montagens mitogenômicas em computadores domésticos, pretendemos montar e anotar pelo menos 100 sequências mitocondriais de insetos usando dados públicos (chamado “projeto 100 MITO”). Trata-se de um projeto de escopo bem maior se comparado àquele descrito nessa dissertação, que consequentemente envolve grande parte dos integrantes do Laboratório de Genômica e Biodiversidade.
Até o momento, possuímos 40 mitogenomas completos ou quase completos (aqui definidos como sequências que, apesar de não circularizadas, apresentam todas as 37 features mitocondriais). Destes, 34 pertencem à espécies de um gênero de formigas hiperdiverso (Temnothorax spp.), montadas a partir dos datasets de UCE gerados pelo trabalho de Prebus (2017). Pretendemos continuar usando os dados públicos disponíveis no Sequence Read Archive para aumentar a cobertura mitogenômica e desvendar relações evolutivas não só de formigas, como também de outros grupos de insetos.
 
