\documentclass[../DISSERTACAO_MAIN.tex]{subfiles}

\begin{document}
	
	Neste estudo, usamos dados públicos para montar, anotar, comparar e fornecer análises evolutivas de 14 sequências completas de genoma mitocondrial da subfamília Pseudomyrmecinae e outros 15 mitogenomas de formigas baixados do GenBank.
	
	\section{Cobertura uniforme do mitogenoma e viés de AT}
	
	Mesmo que segmentos do genoma mitocondrial possam ser copiados para o núcleo formando NuMTs (Sequências Nucleares Mitocondriais), a cobertura genômica obtida para as montagens frequentemente apresentou distribuições uniformes (Figura S1), mesmo para Pseudomyrmex gracilis em que os NuMTs foram previamente identificados (Rubin \& Moreau, 2016 ). A montagem correta dos genomas mitocondriais foi possível porque o número de reads mitocondriais é provavelmente muito maior do que o número de sequências provenientes de NuMTs.
	
	A baixa cobertura em segmentos com uma tendência a AT pronunciada deve ser esperada pois as regiões ricas em AT são conhecidas por ter amplificação reduzida em protocolos de preparação de bibliotecas Illumina (Dohm et al., 2008; Aird et al., 2011; Oyola et al., 2012). Nas espécies analisadas, a região intergênica na qual o D-loop se encontra está localizada entre o RNA ribossomal 12S (rrns) e o RNA transportador da metionina (trn-M) e varia em tamanho de 527 pb (P. concolor) a 697 pb (P. elongatus) (Tabela S1). Vários estudos demonstraram que formigas apresentam viés de AT em seus mitogenomas, em especial na região de controle, que pode exceder 90 \% (Berman et al., 2014; Liu et al., 2016). Nossos dados corroboram isso, já que a menor porcentagem de AT dentre as 14 espécies de Pseudomyrmecinae analisadas corresponde a 74 \% e 12 dessas espécies apresentam valor de conteúdo de AT igual ou superior a 90 \% na região intergênica que contém o D-loop (Tabela 1). Além disso, a região de controle já se mostrou particularmente difícil de sequenciar em himenópteros (Castro \& Dowton, 2005; Dowton et al., 2009; Rodovalho et al., 2014). Tendo isso em vista, o fato de que segmentos de baixa cobertura em nossas montagens sempre ocorrerem no D-loop provavelmente está associado ao viés de AT pronunciado dessa região.
	
	O viés de AT dos mitogenomas de formigas, associado às limitações do sequenciamento para regiões ricas nesses nucleotídeos podem dificultar a obtenção de mitogenomas completos de formigas. Essa dificuldade provavelmente é parte do motivo pelo qual há tão poucos genomas mitocondriais completos disponíveis para esse grupo apesar da ampla disponibilidade de dados públicos. Ao mesmo tempo, devemos considerar que o advento de novas ferramentas podem tornar a montagem de mitogenomas mais acessível. Por exemplo, novos montadores como o NOVOPlasty superam programas clássicos (Dierckxsens, Mardulyn \& Smits, 2016; Plese et al., 2018) e facilitam a produção de mitogenomas completos. Assim sendo, os incessantes avanços técnicos no ramo da bioinformática prenunciam perspectivas favoráveis para o fechamento de lacunas filogenéticas em Formicidae, especialmente se os dados públicos disponíveis para o clado forem utilizados para esse fim.
	
	\section{Mitogenômica Comparativa: tamanho do mitogenoma e análises de sintenia}
	
	Além da identificação de quatro sítios putativos de inserção que podem explicar as diferenças observadas no tamanho do mitogenoma (apontado pelas setas na Figura 8), também observamos que todos os sete mitogenomas incluídos no grupo de espécies P. ferrugineus têm aproximadamente o mesmo tamanho de sequência em pb, sugerindo que esse grupo é monofilético. Por outro lado, existe uma diferença significativa entre o tamanho do mitogenoma de P. concolor (15906 pb) e P. dendroicus (17362 pb), ambas pertencentes ao grupo P. viidus. Isto corrobora trabalhos anteriores que apontam este grupo de espécies como parafilético (Ward, 1989; Ward \& Downie, 2005).
	
	Há uma correlação positiva entre as múltiplas sintenias encontradas nos clados Myrmicinae e Formicinae e a notável biodiversidade observada para estas duas subfamílias: Myrmicinae, que apresentou o maior número de rearranjos gênicos (três), é a maior subfamília de formigas em termos de riqueza de espécies, com mais de 6.600 espécies descritas, quase metade de toda a biodiversidade documentada para formigas; e Formicidae, que apresentou duas sintenias distintas, é a segunda mais biodiversa, com mais de 3.100 espécies. As outras subfamílias de Formicidae analisadas neste estudo (ambas com um único arranjo gênico) são menos diversas: Dolichoderinae exibe ~ 713 espécies enquanto Pseudomyrmecinae apresenta ~ 231 espécies documentadas (Bolton, 2012). Como o arranjo de genes ancestrais para Formicinae é idêntico ao observado em Pseudomyrmecinae e Dolichoderinae, a análise de sintenia indica que Formicinae está mais próximo filogeneticamente a este grupo do que a Myrmicinae. Um número maior de mitogenomas associado a uma cobertura taxonômica mais ampla melhorarão a avaliação da correlação entre a ordem dos genes mitocondriais e a biodiversidade da subfamília, permitindo um melhor entendimento da evolução mitocondrial da sintenia em Formicidae.
	
	
	
	\section{Relações filogenômicas de Formicidae inferidas usando dados de mitogenoma}
	
	As árvores filogenômicas geradas forneceram topologias ligeiramente diferentes devido a informação adicional presente na análise do mitogenoma completo. Enquanto a árvore de concatenação gênica utiliza apenas a informação contida nos genes codificadores de proteínas, a árvore construída com base na sequência mitocondrial completa usa, além das PCG’s, informação proveniente das regiões intergênicas e dos tRNAs, rRNAs e D-loop para a inferência filogenética. Ademais, o DNA mitocondrial apresenta uma taxa de substituição relativamente elevada em regiões não codificantes (Vanecek, Vorel \& Sip, 2004; DeSalle, 2017) e a adição dessa variabilidade às análises também justifica as diferenças topológicas observadas.
	
	Em geral, nas árvores filogenômicas geradas para todas as formigas apresentando mitogenoma completo, a filogenia da subfamília Pseudomyrmecinae foi fortemente recuperada como monofilética, e as posições filogenéticas da maioria dos clados foram bem resolvidas. A monofilia para a subfamília Pseudomyrmecinae e também para os gêneros Pseudomyrmex e Tetraponera foi recuperada com 100 \% de suporte de bootstrap (BS) em ambas as árvores. O gênero Pseudomyrmex apresentou poucos nós não suportados, mas Tetraponera foi completamente resolvida em ambas as árvores (BS = 100). Com relação aos grupos de espécies em Pseudomyrmex, em ambas as árvores o status monofilético do grupo P. flavicornis e o estado parafilético do grupo P. viidus confirmam (i) observações prévias baseadas exclusivamente em morfologia (Ward, 1989), (ii) filogenias usando caracteres morfológicos em conjunto com marcadores nucleares (Ward \& Downie, 2005), e (iii) nossas próprias observações em relação ao tamanho do mitogenoma. Embora a divisão morfológica em grupos de espécies não tenha sido formalizada ou regulada nomenclaturalmente (Ward, 2017), o trabalho usando uma abordagem híbrida morfológica/molecular de Ward \& Downie, 2005 mostra que apenas dois dos nove grupos definidos na época eram parafiléticos: P. pallens e P. viidus. A corroboração de estudos morfológicos pela análise de dados mitocondriais confirma a relevância do uso de caracteres morfológicos na determinação das relações entre clados. Ao mesmo tempo, nossos resultados reforçam que evidências moleculares podem esclarecer e complementar tais estudos, refinando e melhorando o suporte geral das filogenias reconstruídas. Neste trabalho, geramos sequências mitocondriais completas para formigas classificadas em cinco dos 10 grupos descritos para espécies de Pseudomyrmex, cobrindo pelo menos metade da diversidade genética do gênero e adicionando uma nova fonte de evidência molecular para estudos posteriores sobre o clado.
	
	Ambas as árvores sugerem fortemente que os mutualismos de formigas são parafiléticos em Pseudomyrmecinae (por favor, verifique os asteriscos presentes nas Figuras 10 e 11), adicionando também evidências às suposições prévias de comportamento generalista como um traço basal do gênero Pseudomyrmex (Ward \& Branstetter, 2017). Isso sugere que a relação de co-evolução entre plantas e essas formigas se desenvolveu mais tarde (e independentemente) várias vezes no clado. Espécies mutualistas são mais comuns no grupo de espécies P. ferrugineus, reforçando a hipótese do mutualismo ser uma característica derivada. No gênero Pseudomyrmex, o grupo P. ferrugineus possivelmente apresenta duas linhagens independentes de formigas mutualísticas (já que P. feralis é frequentemente considerada como exibindo comportamento generalista; BS = 50), enquanto outras duas linhagens mutualistas independentes podem ser observadas dentro do gênero ao se analisar o posicionamento filogenético de P. concolor e P. dendroicus nas árvores. Considerando o gênero Tetraponera, T. aethiops e T. rufonigra são espécies intimamente relacionadas e apenas T. aethiops apresenta comportamento mutualístico (Ward \& Downie, 2005), mostrando-nos que a diferenciação de traços ecológicos pode ser observado mesmo entre espécies aparentadas. Infelizmente não há estudos que estimem o tempo de divergência para espécies do gênero Tetraponera, mas essa diferenciação ecológica observada em espécies próximas (para as quais é esperado um passado evolutivo comum relativamente recente) pode ser indicativa de que a evolução de traços mutualistas em Pseudomyrmecinae pode ocorrer em períodos de tempo consideravelmente curtos. Considerando o número limitado de espécies amostradas aqui, fomos capazes de identificar 5 das 12 vezes em que associações mutualísticas desenvolvidas independentemente foram relatadas no clado (Ward, 1991; Ward \& Downie, 2005). Com uma melhor cobertura taxonômica, esse número pode ser aumentado e novas análises realizadas, gerando resultados mais robustos e elucidando cada vez mais esses eventos coevolutivos.
	
	Há diversas relações filogenéticas bem resolvidas para várias espécies de Pseudomyrmecinae (como P. peperi, P. veneficus, P. particeps, P. gracilis, T. aethiops e T. rufonigra) que corroboram tanto os resultados de Ward \& Downie (2005) quanto a árvore de Máxima Verossimilhança gerada usando dados de elementos ultra-conservados de Ward \& Branstetter (2017). A relação do grupo irmão entre P. dendroicus e P. elongatus também é bem suportada (BS = 100 na árvore de mitocôndria completa; e BS = 99 na árvore de concatenação gênica), indo ao encontro de um trabalho recente utilizando scaffolds de WGS concatenados como entrada para a construção de árvores de ML (Rubin \& Moreau, 2016).
	
	No entanto, diferenças sutis foram observadas entre nossos resultados e as relações filogenéticas inferidas com base em elementos ultra-conservados de Ward \& Branstetter, 2017. Usando dados de UCE, P. janzeni foi observado como grupo irmão de P. ferrugineus. Neste trabalho, a árvore, usando a sequência mitogenômica completa, recapturou essa mesma relação com um valor replicado de bootstrap de 77. Por outro lado, na árvore de genes concatenados, além da relação de grupo irmão ter sido observada entre P. janzeni e P. flavicornis, ela mostrou um suporte inferior (BS = 47). No geral, essa relação pareceu ser melhor reconstruída pela análise da sequência mitocondrial completa, que corrobora as análises de UCE.
	Dentro de Pseudomyrmecinae, observamos duas espécies cujas posições filogenéticas não foram bem resolvidas pelas análises mitocondriais atuais e, portanto, sua relação pode ser vista como inconclusiva: (i) P. feralis em ambas as árvores filogenéticas; e (ii) P. pallidus na árvore de genes concatenados. Essas posições apresentam um valor de suporte de bootstrap de 50.
	
	Ambas as árvores apresentaram a subfamília Dolichoderinae como monofilética, embora este resultado não tenha sido recuperado em todas as replicatas. Dolichoderinae é uma subfamília altamente diversificada e contém mais de 700 espécies, mas foi aqui representada por apenas duas espécies. Assim, acreditamos que uma maior cobertura de espécies irá melhorar a robustez das análises filogenéticas para o clado.
	
	Trabalhos anteriores com caracteres morfológicos e/ou genes nucleares apresentam evidências de relação de grupo irmão entre Pseudomyrmecinae e Myrmeciinae (Ward \& Downie, 2005; Brady et al., 2006). Nós esperaríamos que Myrmeciinae fosse o grupo irmão de Pseudomyrmecinae de acordo com os dados mitocondriais, mas como genomas mitocondriais completos não estão disponíveis para a subfamília Myrmeciinae, nós não pudemos testar essa hipótese. A ausência de mitogenomas para essa e outras subfamílias podem estar associadas ao fato de que sua biodiversidade e importância econômica não são tão expressivas quando comparadas à das subfamílias de formigas mais estudadas. Por exemplo, Myrmeciinae não apresenta nenhuma espécie de pronunciada relevância econômica e conta com apenas 94 espécies descritas, enquanto Myrmicinae engloba as saúvas (tribo Attini), notoriamente conhecidas como importantes pragas agrícolas e possui mais de 6600 espécies (Bolton, 2012).
	
	Na ausência de Myrmeciinae, espera-se que Dolichoderinae seja o grupo mais próximo de Pseudomyrmecinae em nossas árvores. Isso foi confirmado em ambas as árvores, nas quais as duas subfamílias aparecem como grupos irmãos entre si, corroborando filogenias moleculares de larga escala usando poucos genes nucleares (Brady et al., 2006) e dados de UCE (Branstetter et al., 2017). A sintenia compartilhada entre todas as espécies de Pseudomyrmecinae e Dolichodrinae também suporta a relação do grupo irmão observada. Nossos resultados sugerem que Myrmicinae é o táxon mais próximo de um clado contendo Pseudomyrmecinae e Dolichoderinae, enquanto Formicinae foi observado como um grupo mais basal na família Formicidae. Esta posição basal de Formicinae é altamente suportada na árvore de concatenação de genes, mas não na árvore usando mitogenomas completos, ao contrário do que é mostrado por outros trabalhos usando dados nucleares que apontam para uma relação de grupo irmão entre Myrmicinae e Formicinae (Brady et al., 2006; Branstetter et al., 2017).
	
	A monofilia da subfamília Formicinae e todos os seus nós mostram suporte máximo em ambas as árvores (BS = 100). Nossos resultados também corroboram o caráter monofilético do gênero Formica e apresentam os gêneros Camponotus e Polyrhachis como intimamente relacionados entre si, conforme observado no trabalho de Blaimer e colaboradores (2015), que utilizaram locus de UCE para inferência filogenética. O único problema com relação a essa subfamília diz respeito à posição mal suportada de Formicinae em relação às outras subfamílias. Os dados de mitogenoma forneceram com sucesso relações filogenéticas robustas, mesmo para Camponotus atrox, uma espécie que mostrou sintenia única, mas teve sua posição bem resolvida em ambas as inferências, inclusive na árvore mitocondrial completa, que pode estar propensa a ser afetada por alterações de sintenia. Esta questão confirma a robustez das sequências mitocondriais para inferir filogenias de formigas.
	
	No geral, os resultados mais controversos obtidos aqui estão relacionados à posição da subfamília Myrmicinae. Para esse clado, a árvore de concatenação de genes foi capaz de indicar monofilia (BS = 74), mas dados de mitogenoma total produziram parafilia. Neste último caso, as espécies Atta texana, Myrmica scabrinodis e Pristomyrmex punctatus divergiram antes das outras formigas. Por outro lado, ambas as árvores recapturaram com sucesso o caráter monofilético do gênero Solenopsis e as relações entre suas espécies (S. geminata como grupo irmão do clado consistindo de S. invicta e S. richteri) com 100 \% de suporte de bootstrap. A relação do grupo irmão entre Solenopsis spp. e Vollenhovia emeryi também é recuperada. Estes resultados corroboram aqueles obtidos pelo uso de sequências de aminoácidos concatenados de todos os PCGs mitocondriais para inferência de árvores (Duan et al., 2016). Entretanto, nossa avaliação da posição de V. emeryi foi melhor suportada (BS = 90 na árvore de concatenação gênica e BS = 99 na árvore mitocondrial completa) do que a deste trabalho anterior (BS = 75). Considerando que Duan e colaboradores (2016) utilizaram uma abordagem semelhante à nossa (concatenação gênica seguida por construção de árvore usando Máxima Verossimilhança), podemos concluir que esses resultados indicam que os dados nucleotídicos apresentam informações mais confiáveis para a inferência filogenômica desses clados do que os dados aminoacídicos. Isto é consistente com pesquisa anterior na qual a inferência filogenética utilizando nucleotídeos obteve resultados melhor suportados do que as análises ao nível de aminoácidos ou codons (Holder, Zwickl \& Dessimoz, 2008). Além disso, os valores de bootstrap obtidos através de dados nucleotídicos já foram relatados como geralmente maiores do que aqueles provenientes de seus correspondentes em aminoácidos (Regier et al., 2010). Essas observações são ao menos parcialmente explicadas pelas diferenças na quantidade de sinal filogenético considerados por esses dois métodos. Sinal adicional presente em sequências de nucleotídeos é perdido na tradução para aminoácidos. Isso é particularmente importante em se tratando de aminoácidos hexacodônicos como a serina, que é codificada tanto por TCN quanto por AGY (Regier et al., 2010; Zwick, Regier \& Zwickl, 2012).
	
	As relações filogenéticas de outras espécies da subfamília Myrmicinae na nossa árvore de concatenação gênica não estão bem resolvidas, como a posição de Myrmica scabrinodis (BS = 52), Wasmannia auropunctata (BS = 43) e Pristomyrmex punctatus (BS = 16). No entanto, a posição dessas espécies na árvore de aminoácidos de Duan e colaboradores (2016) também é inconclusiva e difere daqui, agrupando W. auropunctata e M. scabrinodis em uma relação suportada apenas em 35 \% das replicatas de bootstrap. Este clado é colocado como grupo irmão de Solenopsis spp. e V. emeryi com suporte ainda menor (BS = 21) e P. punctatus assume uma posição mais basal na árvore em 46 \% das repetições. No entanto, Atta laevigata aparece na base de todas as Myrmecinae com suporte de bootstrap de 100 \% na árvore de aminoácidos. Como o mitogenoma de A. laevigata disponível não está completo, ele não foi usado como entrada para a concatenação de nucleotídeos das PCGs aqui realizada, ao contrário de sua congenérica Atta texana, cujo mitogenoma completo foi analisado aqui. A. texana também aparece na base da subfamília Myrmicinae, mas sob uma relação de grupo irmão com M. scabrinodis, mesmo que com baixa resolução (BS = 52). Este clado é irmão de todas as outras espécies de Myrmecinae (BS = 74). Por fim, a posição de Cardiocondyla obscurior também não foi bem suportada (BS = 43), mas como esse é um mitogenoma recentemente publicado, não foi utilizado no trabalho de Duan e colaboradores.
	
	Em ambos os trabalhos, as análises mitogenômicas não foram totalmente capazes de resolver importantes nós do ramo das Myrmicinae e vários fatores podem estar associados a esses resultados insatisfatórios. É necessário destacar que Myrmicinae é a subfamília mais biodiversa (Bolton, 2012) e é conhecida por apresentar vários grupos monofiléticos duvidosos (Brady et al., 2006; Ward, 2011; Ward et al., 2015). Essa diversidade é evidenciada pelo fato de que, apesar de apenas nove mitogenomas estarem disponíveis para o grupo, três arranjos diferentes de genes mitocondriais podem ser observados, sugerindo uma alta taxa de evolução mitocondrial nessa subfamília.
	
	Além disso, houveram divergências no ramo das Myrmicinae em estudos filogenéticos moleculares anteriores que tentaram estudar a família Formicidae como um todo (Brady et al., 2006; Moreau et al., 2006). Por outro lado, \citeonline{Ward2015} se foca no estudo dessa subfamília ao reconstruir uma filogenia em grande escala usando 11 marcadores nucleares de 251 espécies amostradas em todas as 25 tribos de Myrmicinae, a maioria delas parafiléticas. Utilizando uma grande quantidade de dados que cobre uma extensa parcela da diversidade de espécies dessa subfamília, eles conseguiram propor uma nova classificação de Myrmicinae composta exclusivamente por tribos monofiléticas, o que também reduziu o número de gêneros parafiléticos.
	
	Assim, a natureza hiperdiversa deste clado, associada à subamostragem ou mesmo ausência de mitogenomas para vários táxons da subfamília e uma possível alta taxa de evolução do genoma mitocondrial são fatores que podem ter contribuído para os resultados inconclusivos das análises mitocondriais. Além disso, apesar de algumas relações não terem sido elucidadas pelo uso exclusivo da filogenômica mitocondrial, a informação fornecida pelo mitogenoma é classicamente considerada como útil no estudo das relações evolutivas para diversos táxons, seja confirmando (Prosdocimi et al., 2012; Finstermeier et al. al., 2013) ou refutando hipóteses filogenéticas anteriores \cite{Kayal2015, Uliano-Silva2016}. Assim, ainda recomendamos o uso de dados mitocondriais, de preferência ao lado de outros marcadores (por exemplo, genes nucleares), para aumentar o sinal filogenético e recapturar filogenias mais robustas. Entretanto, graças à taxa de substituição do mtDNA, árvores geradas a partir de dados mitocondriais apresentam uma maior probabilidade de resolver ramos curtos corretamente (DeSalle, 2017). Portanto, também acreditamos que o uso de dados mitocondriais para inferência filogenômica, mesmo sem outros marcadores, produzirá resultados mais satisfatórios se trabalharmos no sentido de mitigar o problema da escassez de mitogenomas disponíveis para esse clado e melhorarmos a cobertura mitocondrial de seus táxons. Essa afirmativa não só é válida para a subfamília Myrmicinae, como também para a família Formicidae como um todo e para qualquer outro grupo com escassez de mitogenomas conhecidos e necessidade de elucidação sobre suas relações filogenéticas. Nesse sentido, os resultados aqui apresentados são extremamente relevantes para mostrar que as informações já disponíveis em bancos de dados públicos devem ser usadas para obter genomas mitocondriais completos e fomentar novas pesquisas que gerarão conhecimento sem incorrer em custos adicionais de sequenciamento.
	
	\section{Mitogenômica no-budget: análises integradas entre datasets e potencial para estudos de larga-escala}
	
	Os resultados aqui apresentados confirmam que os dados de UCE e WGS publicamente disponíveis podem ser usados para montar genomas mitocondriais completos com alta cobertura (Tabela 2), o que pode ser explicado pelo alto número de cópias de reads mitocondriais que pode alcançar algo entre 0,25 \% a 0,5 \% do número total de bases geradas \cite{Prosdocimi2012} e chegar a 2 \% do total de reads mapeando ao mtDNA \cite{Ekblom2014}. Também confirmamos o potencial dos dados de UCE como uma alternativa de baixo custo para sequenciar mitogenomas completos com alta cobertura, conforme descrito por Raposo do Amaral et al. (2015). Dados de mitogenoma são usados em várias análises e seqüências mitocondriais são encontradas em vários tipos de datasets, que geralmente fornecem informação suficiente para montar toda a sequência mitocondrial. Essa versatilidade e onipresença de sequências mitocondriais deve ser usada em favor dos estudos de biodiversidade, especialmente considerando que os datasets públicos estão disponíveis para um número cada vez maior de espécies.
	
	O potencial dessas seqüências na elucidação de filogenias não deve ser menosprezado, especialmente se considerarmos que existem diferentes tipos de conjuntos de dados disponíveis para diferentes espécies (WGS, RNA-Seq, enriquecimento de UCE, dentre outros). Esses diferentes recursos dificultam a obtenção de árvores filogenéticas/filogenômicas que integrem esses diferentes dados públicos, já que muitas vezes as análises dependem da ortologia das sequências comparadas \cite{Kuzniar2008}. Assim, o uso de diferentes tipos de dados para montar os mitogenomas completos ou quase completos para espécies com dados publicamente disponíveis apresenta uma solução para este problema com o genoma mitocondrial agindo como uma “sequência normalizadora” que permite a comparação de diferentes conjuntos de dados. Por exemplo, neste trabalho algumas espécies tinham apenas dados de UCE disponíveis publicamente, enquanto outros apresentavam datasets padrão de WGS. No entanto, anotando e analisando o mitogenoma completo para essas espécies, conseguimos ampliar nosso escopo e estudar todas elas juntas. Assim, sugerimos que o uso de mitogenomas obtidos a partir de dados públicos tem o potencial de se tornar uma importante fonte de informação filogenética. Além disso, o estudo das sequências mitocondriais pode ser uma das rotas mais rápidas para a obtenção de árvores abrangentes de alta qualidade para táxons hiperdiversos, como os insetos. Progresso se tem feito nesse sentido, como pode ser visto no trabalho recente de Linard et al. (2018), onde a mineração de dados do Genbank e montagem usando datasets metagenômicos forneceram contigs mitocondriais (> 3kpb) para quase 16.000 espécies de coleópteros. Essa enorme quantidade de dados mitogenômicos foi usada para gerar a maior árvore filogenética já vista para o clado.
	
	Estudos que tentam montar mitogenomas completos usando dados públicos ainda são escassos, ao passo que o tamanho e a amplitude dos bancos de dados públicos estão em crescimento, juntamente com seu potencial para responder questões filogenéticas, dentre tantas outras. A mitogenômica no-budget é uma oportunidade sem precedentes de reconstruir e analisar filogenias em larga escala para vários grupos em diferentes níveis taxonômicos, o que por sua vez pode subsidiar estudos evolutivos e de biologia da conservação e incrementar nosso conhecimento sobre espécies não-modelo e sua diversidade.
	
\end{document}